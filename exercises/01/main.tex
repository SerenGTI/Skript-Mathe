\documentclass[a4paper, oneside]{article}
\usepackage[utf8]{inputenc}
\usepackage[ngerman]{babel}
\usepackage[top=2.5cm, bottom=3cm, outer=2.5cm, inner=2.5cm, heightrounded]{geometry}
\usepackage{graphicx}
\usepackage{morefloats}
\usepackage{wrapfig}
\usepackage{hyperref}
\usepackage{cite}
\usepackage{siunitx}
\usepackage[default]{sourcesanspro}
\usepackage[T1]{fontenc}
\usepackage{url}
\usepackage{marginnote}
\usepackage[font=footnotesize]{caption}
\usepackage{color}
\usepackage{xcolor}
\usepackage{multicol}
\usepackage[fleqn]{mathtools}
\usepackage{amssymb}
\usepackage{wrapfig}
\usepackage[noindentafter]{titlesec}
\usepackage{fancyhdr}
\usepackage{lastpage}
\usepackage{comment}

%% LÖSUNGEN ANZEIGEN
\newif\ifshow
%\showtrue
\showfalse

%%%SECTIONING
\renewcommand*{\marginfont}{\noindent\rule{0pt}{0.7\baselineskip}\footnotesize}

\newcommand{\aufgabe}[1]{\subsection{#1}}
\newcommand{\loesung}[1]{\subsubsection{#1}}

\newcommand{\simpleset}[1]{\ensuremath \left\{ #1 \right\}}
\newcommand{\ematrix}[2]{\renewcommand{\arraystretch}{1}\ensuremath\left(\begin{array}{@{}#1@{}}#2\end{array}\right)}

\renewcommand{\theenumi}{\alph{enumi})}
\renewcommand{\labelenumi}{\text{\theenumi}}

\newcounter{aufgabe}
%\newenvironment{lsg}{\loesung}{}
\ifshow
  \newenvironment{lsg}{\loesung}{}
\else
  \excludecomment{lsg}
\fi

\newenvironment{inhalt}
  {\paragraph{Inhalt des Übungsblatts:}\itemize\let\origitem\item}
  {\enditemize\vspace{2em}}

\newcommand{\R}{\ensuremath\mathbb{R}}
\newcommand{\N}{\ensuremath\mathbb{N}}
\newcommand{\Z}{\ensuremath\mathbb{Z}}
\newcommand{\LM}{\ensuremath\mathbb{L}}
\newcommand{\intd}{\ensuremath\mathrm{d}}
\newcommand{\e}{\ensuremath\mathrm{e}}
\renewcommand{\d}{\,\mathrm{d}}
\newcommand{\stf}[1]{\ensuremath \left[ #1 \right]}

\newcommand{\cas}{\hfill (CAS)}
\newcommand{\seite}[1]{\textit{(S. #1)}}

\newcommand{\vektor}[1]{\ensuremath\begin{pmatrix} #1 \end{pmatrix}}


\everymath{\displaystyle}

%Malpunkte
\mathcode`\*="8000
{\catcode`\*\active\gdef*{\cdot}}

%SECTION
\titleformat{\section}
{\clearpage\setcounter{aufgabe}{0}\vspace{1em}\Large\raggedright\bfseries}
{}
{0pt}
{}

\titleformat{\subsection}[runin]
{\stepcounter{aufgabe}\vspace{1px}\normalfont\raggedright\bfseries}
{A\theaufgabe: }
{0pt}
{\ }

\titleformat{\subsubsection}[runin]
{\normalfont\raggedright\bfseries}
{Lösung \theaufgabe: }
{0pt}
{\ }


%FANCYHDR
\pagestyle{fancy}
\lhead{\small Simon König\\ Joshua Fabian}
\rhead{\small Mathecrashkurs 2018}
\cfoot{Seite \thepage\thinspace von\thinspace\pageref{LastPage}}
\lfoot{}
\renewcommand{\headrulewidth}{0.5pt}
\renewcommand{\footrulewidth}{0pt}

\title{Mathe-Crashkurs 2018 - Übungsblatt}
\date{\today}
\author{Simon König, Joshua Fabian}

\chead{\Large Übungsblatt 1}

\begin{document}
\begin{inhalt}
	\item Extremstellen und -Punkte \seite{27}
	\item Exponentialfunktion \seite{31}
	\item Funktionenscharen \seite{33}
  \item Integral \seite{37}, Rotationskörper \seite{40}, Flächeninhalte \seite{41}
	\item Funktionsanalyse \seite{45}, gebrochenrationale Funktionen, Asymptoten \seite{46}
\end{inhalt}

\aufgabe{Gleichung lösen}
Löse die Gleichung $\e^{5x}-\e^{3x}=6\e^x$.\\
\textit{Hinweis: Du brauchst ungefähr alle gelernten Methoden!}
\begin{lsg}{}
	\begin{multicols}{2}
	\begin{align*}
		\e^{5x}-\e^{3x}&=6\e^x &&|-6\e^x\\
		\e^{5x}-\e^{3x}-6\e^x&=0 &&|\ \text{Ausklammern}\\
		\e^x\cdot(\e^{4x}-\e^{2x}-6)&=0 &&|\ \text{Nullprodukt und Substitution: } z=\e^{2x}\\
		z^2-z-6&=0 &&|\ \text{Mitternachtsformel}\\
		z_1=-2 \quad z_2&=3&&|\ \text{Resubstitution: } z_{1/2}=\e^{2\cdot x_{1/2}}\\
		\e^{2\cdot x_1}&=-2 &&\Rightarrow \text{nicht möglich ($\e^x$ immer $>0$)}\\
		\e^{2\cdot x_2}&= 3 &&\rightsquigarrow x=\frac{\ln(3)}{2}
	\end{align*}
	\end{multicols}
\end{lsg}


\aufgabe{Exponentialfunktion}
\begin{enumerate}
	\item Gib $f(x)=25^x$ als natürliche Exponentialfunktion an.
	\item	Wie unterscheidet sich der Graph von $-\e^{-x}$ von $\e^x$? Formuliere die Erklärung schrittweise.

\end{enumerate}
\begin{lsg}{}
	\begin{enumerate}
		\item $\e^{\ln(25)\cdot x}$
		\item 	\begin{itemize}
					\item $-\e^x$ ist zu $ \e^x$ an der x-Achse gespiegelt
					\item $\e^{-x}$ ist zu $\e^x$ an der y-Achse gespiegelt
				\end{itemize}
				$\Rightarrow -\e^{-x}$ ist zu $\e^x$ an der x- und der y-Achse gespiegelt
	\end{enumerate}
\end{lsg}

\aufgabe{\color{red}Extrempunkte}
\begin{enumerate}
	\item Bestimme Extrempunkte und Wendestellen von: $f(x)=-x^3+3x^2+5x+7$

	\item Gegeben ist $g(x)=10x*\e^{\frac{1}{2}x}$ (vgl. Abi 2014)\begin{itemize}
		\item Bestimme Extrempunkt und Wendepunkt von $g$
		\item Für jedes $u>0$ sind O$(0|0)$, P$(u|0)$ und Q$(u|g(u))$ die Eckpunkte eines Dreiecks. Bestimme einen Wert für $u$ so, dass dieses Dreieck den Flächeninhalt $8$ hat.
		\item Auf der x-Achse gibt es Intervalle der Länge 3, auf denen die Funktion $g$ den Mittelwert $2,2$ besitzt. Bestimme die Grenzen eines solchen Intervalls.
	\end{itemize}

\end{enumerate}

\begin{lsg}{}
\begin{enumerate}
	\item $f'(x)=-3x^2+6x+5$ und
	$f''(x)=-6x+6$\\
	Nullstellen von $f(x)$ sind
\end{enumerate}
\end{lsg}


\aufgabe{Funktionenscharen: }
\begin{enumerate}
  \item Berechne die Nullstellen der Funktionenscharen in Abhängigkeit von $a\in \R, a\neq 0$:
  \begin{multicols}{2}
  \begin{itemize}
    \item $f_a(x)=x^2+2ax+9$
    \item $g_a(x)=5ax+15a$
    \item $h_a(x)=x^3-a^2$
    \item $j_a(x)=(x-3a)(x+6a)$
  \end{itemize}
  \end{multicols}
  \item Gegeben ist die Funktionenschar $f_a$ mit $f_a(x)=(x+a)*\e^{-x}\ , x\in \R$ .
  \begin{itemize}
    \item Untersuche die Lage des Maxmimums.
    \item Gib die Gleichung der Funktion an, auf der die Maxima aller Scharkurven liegen.
  \end{itemize}
\end{enumerate}
\begin{lsg}{}
	\begin{enumerate}
		\item \begin{itemize}
			\item $0=x^2+2ax+9 \rightsquigarrow x_1=\sqrt{a^2-9}-a,\ x_2=-\sqrt{a^2-9}-a$\\
			\item $0=5ax+15a \rightsquigarrow x=-3$\\
			\item $0=x^3-a^2 \rightsquigarrow x=\sqrt[3]{a^2}$\\
			\item $\text{Nullprodukt:}\ x_1=3a,\ x_2=-6a$
		\end{itemize}
		\item \begin{itemize}
			\item Ableitung bilden: \begin{align*}
				f'_a(x)&=\e^{-x}-x\e^{-x}-a\e^{-x}=(1-x-a)*\e^{-x}\\
				f''_a(x)&=-\e^{-x}-\e^{-x}+x\e^{-x}+a\e^{-x}=(a+x-2)*\e^{-x}
			\end{align*}\\
			Extremstelle: $f'_a(x)=0 \rightsquigarrow x=1-a$ \\
			In $f''_a$ einsetzen: $f''_a(1-a)=(a+1-a-2)*\e^{-(1-a)}<0$ für alle $a\in \R\Rightarrow $HP bei $x=1-a$\\
			HP bei $(1-a|\e^{a-1})$
			\item Aus dem HP bei $(1-a|\e^{a-1})$ folgt $x_{h}=1-a \rightsquigarrow a=1-x_{h}$\\
			Eingesetzt in $y_{h}=\e^{a-1}$ folgt: $y_h=\e^{-x_h}$ \\
			Damit gilt für die Ortskurve: $t(x)=\e^{-x}$

		\end{itemize}

	\end{enumerate}
\end{lsg}



\aufgabe{Stammfunktion berechnen: } Berechne jeweils ein Stammfunktion zu den angegebenen Funktionen:
\begin{multicols}{3}
	\begin{enumerate}
		\item $f(x)=x^2+x-3$
		\item $g(x)=(2x-3)^8$
		\item $h(x)=-5\sin(3x+2)$
		\item $i(x)=\e^{3x+7}$
		\item $j(x)=\frac{1}{x*\ln x}$
		\item $k(x)=\e^{x-\e^x}$
	\end{enumerate}
\end{multicols}
\begin{lsg}{}
	\begin{multicols}{2}
		\begin{enumerate}
			\item $F(x)=\frac 1 3x^3+\frac 1 2 x^2-3x$
			\item $G(x)=\frac 1 9 (2x-3)*\frac 1 2=\frac{1}{18}(2x-3)$
			\item $H(x)=\frac 5 3 \cos(3x+2)$
			\item $I(x)=\frac 1 3 \e^{3x+7}$
			\item $j(x)=\frac 1 x*\frac{1}{\ln(x)}\rightsquigarrow J(x)=\ln(\ln x)$
			\item $k(x)=\e^x*\e^{-\e^x}\rightsquigarrow K(x)=-\e^{-\e^x}$
		\end{enumerate}
	\end{multicols}
\end{lsg}




\aufgabe{Integral: }
\begin{enumerate}
	\item Welche der Auswahlmöglichkeiten können eingesetzt werden?
	\begin{equation*}
		\int\limits_0^5 \left(3x^2+ \frac 1 5 x\right)  \d x = \Box
	\end{equation*}
  \begin{multicols}{4}
    \begin{itemize}
      \item $\stf{6x + \frac{1}{5}}_0^5$
      \item $\bigg[x^3 + 0,1 x^2\bigg]_0^5$
  		\item $127,5$
      \item $\stf{x^3 + \frac{1}{10} x^2}_1^6$
  	\end{itemize}
  \end{multicols}
  \item Berechne den Gesamtinhalt der Flächen, die durch die Schaubilder der Funktionen eingeschlossen wird:
  \begin{itemize}
    \item $f(x)=x^2, g(x)=2-x^2$
    \item $h(x)=x^3, i(x)=x^2$
    \item $j(x)=x^3, k(x)=x\ $ (Achtet auf Flächen über und unter der x-Achse)
  \end{itemize}
\end{enumerate}
\begin{lsg}{}
  \begin{enumerate}
    \item $\bigg[x^3 + 0,1 x^2\bigg]_0^5$ und $127,5$
    \item
    \begin{itemize}
      \item Schnittpunkte der Graphen: $x_{1,2}=\pm 1$\\ $\int\limits_{-1}^1 2-x^2-x^2\d x=\left[2x-\frac{2}{3}x^3\right]_{-1}^1=\frac 8 3$
      \item Schnittpunkt der Graphen: $x_1=0,\ x_2=1$ \\$\int_0^1 (x^2-x^3)\d x=\left[\frac{1}{3}x^3-\frac{1}{4}x^4\right]_0^1=\frac 1 {12}$
      \item Schnittpunkte der Graphen: $x_{1,2}=\pm 1,\ x_3=0$ \\$\int_{-1}^0 (x^3-x)\d x+\int_0^1 (x-x^3)\d x=2*\left[\frac{1}{2}x^2-\frac{1}{4}x^4\right]_0^1=\frac 1 2$
    \end{itemize}
  \end{enumerate}
\end{lsg}


\aufgabe{Integral: }
Die Gerade $y=x$ und die $x$-Achse begrenzen zusammen mit den Geraden $x=2$ und $x=u$ mit $u>2$ eine Fläche. Bestimmen Sie einen Wert für $u$ so, dass $f(x)=x-\frac{8}{x^2}$ diese Fläche in zwei inhaltsgleiche Teile zerlegt. \cas
\begin{lsg}{}
  Bestimme die Flächeninhalte der Teilflächen:\begin{align*}
  A_1&=\int_2^u\left(x-\left(x-\frac{8}{x^2}\right)\right)\d x&&|\ \text{Fläche zwischen $y=x$ und $f(x)=x-\frac{8}{x^2}$}\\
  A_2&= \int_2^u\left(x-\frac{8}{x^2}\right)\d x&&|\ \text{Fläche zwischen $f(x)=x-\frac{8}{x^2}$ und $x$-Achse}
  \end{align*}
  Gleichsetzten und mittels CAS nach $u$ auflösen ergibt: $u\approx 3,12$
\end{lsg}


\aufgabe{Uneigentliches Flächenintegral}
\begin{enumerate}
	\item Berechnen Sie $\int\limits^\infty_0 e^{-x}\d x$
\end{enumerate}
\begin{lsg}{}
	\begin{align*}
		&\int\limits^k_0 e^{-x}\d x\\
		&=\left[-\e^{-x}\right]_0^k\\
		&=(-\e^{-k})-(-1)\\
		&=-\e^{-k}+1\\
		&\lim\limits_{k\rightarrow \infty}-\e^{-k}+1\\
		&=0+1=1
	\end{align*}
\end{lsg}


\aufgabe{Rotationskörper: }
\begin{enumerate}
  \item Die Fläche, welche von der $x$-Achse und dem Graphen der Funktionen vollständig eingeschlossen wird, rotiert um die $x$-Achse.
  Berechne den Rauminhalt des entstandenen Körpers.
  \begin{multicols}{2}
  \begin{itemize}
    \item $f(x)=x^2-2x$
    \item $g(x)=\sqrt{x}*(x-2)$
    \item $h(x)=\frac 1 3 x^2-x$
    \item $j(x)=x^2-5x+4$
  \end{itemize}
  \end{multicols}
  \item Die Fläche, welche von den Graphen der Funktionen vollständig eingeschlossen wird, rotiert um die $x$-Achse.
  Berechne den Rauminhalt des entstandenen Körpers.
  \begin{itemize}
    \item $f(x)= -x^2+4, \quad g(x)= x+2$
    \item $h(x)= x^2-x+1, \quad j(x)= 4x-3$
  \end{itemize}
\end{enumerate}
\begin{lsg}{}
	\begin{enumerate}
		\item \begin{itemize}
			\item Nullstellen bei $x_0=0, x_1=2$
			\begin{equation*}
				V=\pi*\int\limits_0^2(x^2-2x)^2\d x =\frac {16\pi}{15}\approx 3,35
			\end{equation*}
			\item Nullstellen bei $x_0=0, x_1=2$
			\begin{equation*}
				V=\pi*\int\limits_0^2(\sqrt{x}*(x-2))^2\d x =\frac {4\pi}{3}\approx 4,18
			\end{equation*}
			\item Nullstellen bei $x_0=0, x_1=3$
			\begin{equation*}
				V=\pi*\int\limits_0^3(\frac 1 3 x^2-x)^2\d x =\frac {9\pi}{10}\approx 2,83
			\end{equation*}
			\item Nullstellen bei $x_0=1, x_1=4$
			\begin{equation*}
				V=\pi*\int\limits_1^4(x^2-5x+4)^2\d x =\frac {81\pi}{10}\approx 25,4
			\end{equation*}
		\end{itemize}
		\item \begin{itemize}
			\item Schnittpunkte bei $x_0=-2, x_1=1$
			\begin{equation*}
				V=\pi*\int\limits_-2^1\left((-x^2+4)-(x+2)\right)^2\d x =\frac {81\pi}{10}\approx 25,4
			\end{equation*}
			\item Schnittpunkte bei $x_0=1, x_1=4$
			\begin{equation*}
				V=\pi*\int\limits_1^4\left((4x-3)-(x^2-x+1)\right)^2\d x =\frac {81\pi}{10}\approx 25,4
			\end{equation*}
		\end{itemize}
	\end{enumerate}
\end{lsg}

\aufgabe{Asymptoten}
\begin{enumerate}
	\item Gib die x- und y-Werte der senkrechten bzw. waagrechten Asymptoten der Funktionen an:
	\begin{multicols}{3}
	\begin{itemize}
		\item $f(x)=\frac{1}{x}$
		\item $g(x)=\frac{1}{x-1}$
		\item $h(x)=\frac{2x}{x^2-1}$
		\item $i(x)=\e^{-x}+1$
		\item $j(x)=-\e^{x}-4$
		\item $k(x)=\frac{x^2+4x-5}{2x^2-4}$
	\end{itemize}
	\end{multicols}
	\item Gib die Gleichung der gebrochenrationalen Funktion $f$ mit folgenden Eigenschaften an:\\
	Asymptoten: $x=-2,\ x=2,\ y=-4$ und Nullstellen: $x=3$
\end{enumerate}
\begin{lsg}{}
	\begin{enumerate}
	\item \begin{multicols}{2}
		\begin{itemize}
			\item $f(x)$: Senkrechte Asymptote bei $x=0$,\\ waagrechte Asymptote bei $y=0$
			\item $g(x)$: Senkrechte Asymptote bei $x=1$,\\ waagrechte Asymptote bei $y=0$
			\item $h(x)$: Senkrechte Asymptote bei $x_{1,2}=\pm 1$,\\ waagrechte Asymptote bei $y=0$
			\item $i(x)$: waagrechte Asymptote bei $y=1$
			\item $j(x)$: waagrechte Asymptote bei $y=-4$
			\item $k(x)$: Senkrechte Asymptote bei $x_{1,2}=\pm \sqrt{2}$,\\ waagrechte Asymptote bei $y=\frac{1}{2}$
		\end{itemize}
		\end{multicols}
		\item senkrechte Asymptoten bei $x=-2,\ x=2$ ergibt: Nenner muss $-2$ und $2$ als Nullstellen haben $\rightsquigarrow x^2-4$\\
		waagrechte Asymptote bei $y=-4$ ergibt: Zähler und Nenner müssen selben Grad haben $\left(x^2\right)$ und als Quotient $-4$ ergeben $\rightsquigarrow \frac{-4x^2}{x^2-4}$\\
		Nullstelle $x=3$ ergibt: Zähler muss $3$ als Nullstelle haben $\rightsquigarrow \frac{-4x^2+36}{x^2-4}$
	\end{enumerate}
\end{lsg}


\end{document}
