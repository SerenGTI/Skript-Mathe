\documentclass[a4paper, oneside]{article}
\usepackage[utf8]{inputenc}
\usepackage[ngerman]{babel}
\usepackage[top=2.5cm, bottom=3cm, outer=2.5cm, inner=2.5cm, heightrounded]{geometry}
\usepackage{graphicx}
\usepackage{morefloats}
\usepackage{wrapfig}
\usepackage{hyperref}
\usepackage{cite}
\usepackage{siunitx}
\usepackage[default]{sourcesanspro}
\usepackage[T1]{fontenc}
\usepackage{url}
\usepackage{marginnote}
\usepackage[font=footnotesize]{caption}
\usepackage{color}
\usepackage{xcolor}
\usepackage{multicol}
\usepackage[fleqn]{mathtools}
\usepackage{amssymb}
\usepackage{wrapfig}
\usepackage[noindentafter]{titlesec}
\usepackage{fancyhdr}
\usepackage{lastpage}
\usepackage{comment}

%% LÖSUNGEN ANZEIGEN
\newif\ifshow
%\showtrue
\showfalse

%%%SECTIONING
\renewcommand*{\marginfont}{\noindent\rule{0pt}{0.7\baselineskip}\footnotesize}

\newcommand{\aufgabe}[1]{\subsection{#1}}
\newcommand{\loesung}[1]{\subsubsection{#1}}

\newcommand{\simpleset}[1]{\ensuremath \left\{ #1 \right\}}
\newcommand{\ematrix}[2]{\renewcommand{\arraystretch}{1}\ensuremath\left(\begin{array}{@{}#1@{}}#2\end{array}\right)}

\renewcommand{\theenumi}{\alph{enumi})}
\renewcommand{\labelenumi}{\text{\theenumi}}

\newcounter{aufgabe}
%\newenvironment{lsg}{\loesung}{}
\ifshow
  \newenvironment{lsg}{\loesung}{}
\else
  \excludecomment{lsg}
\fi

\newenvironment{inhalt}
  {\paragraph{Inhalt des Übungsblatts:}\itemize\let\origitem\item}
  {\enditemize\vspace{2em}}

\newcommand{\R}{\ensuremath\mathbb{R}}
\newcommand{\N}{\ensuremath\mathbb{N}}
\newcommand{\Z}{\ensuremath\mathbb{Z}}
\newcommand{\LM}{\ensuremath\mathbb{L}}
\newcommand{\intd}{\ensuremath\mathrm{d}}
\newcommand{\e}{\ensuremath\mathrm{e}}
\renewcommand{\d}{\,\mathrm{d}}
\newcommand{\stf}[1]{\ensuremath \left[ #1 \right]}

\newcommand{\cas}{\hfill (CAS)}
\newcommand{\seite}[1]{\textit{(S. #1)}}

\newcommand{\vektor}[1]{\ensuremath\begin{pmatrix} #1 \end{pmatrix}}


\everymath{\displaystyle}

%Malpunkte
\mathcode`\*="8000
{\catcode`\*\active\gdef*{\cdot}}

%SECTION
\titleformat{\section}
{\clearpage\setcounter{aufgabe}{0}\vspace{1em}\Large\raggedright\bfseries}
{}
{0pt}
{}

\titleformat{\subsection}[runin]
{\stepcounter{aufgabe}\vspace{1px}\normalfont\raggedright\bfseries}
{A\theaufgabe: }
{0pt}
{\ }

\titleformat{\subsubsection}[runin]
{\normalfont\raggedright\bfseries}
{Lösung \theaufgabe: }
{0pt}
{\ }


%FANCYHDR
\pagestyle{fancy}
\lhead{\small Simon König\\ Joshua Fabian}
\rhead{\small Mathecrashkurs 2018}
\cfoot{Seite \thepage\thinspace von\thinspace\pageref{LastPage}}
\lfoot{}
\renewcommand{\headrulewidth}{0.5pt}
\renewcommand{\footrulewidth}{0pt}

\title{Mathe-Crashkurs 2018 - Übungsblatt}
\date{\today}
\author{Simon König, Joshua Fabian}

\chead{\Large Übungsblatt 1}

\begin{document}
\begin{inhalt}
	\item Extremstellen und -Punkte
	\item Exponentialfunktion, Logarithmus
	\item Funktionenscharen
  \item Integral, Rotationskörper, Flächeninhalte
	\item Funktionsanalyse, gebrochenrationale Funktionen, Asymptoten
\end{inhalt}


\aufgabe{Gleichung lösen}
Löse die Gleichung $\e^{5x}-\e^{3x}=6\e^x$.\\
\textit{Hinweis: Du brauchst ungefähr alle gelernten Methoden!}
\begin{lsg}{}
	\begin{multicols}{2}
	\begin{align*}
		\e^{5x}-\e^{3x}&=6\e^x &&|-6\e^x\\
		\e^{5x}-\e^{3x}-6\e^x&=0 &&|\ \text{Ausklammern}\\
		\e^x\cdot(\e^{4x}-\e^{2x}-6)&=0 &&|\ \text{Nullprodukt und Substitution: } z=\e^{2x}\\
		z^2-z-6&=0 &&|\ \text{Mitternachtsformel}\\
		z_1=-2 \quad z_2&=3&&|\ \text{Resubstitution: } z_{1/2}=\e^{2\cdot x_{1/2}}\\
		\e^{2\cdot x_1}&=-2 &&\Rightarrow \text{nicht möglich ($\e^x$ immer $>0$)}\\
		\e^{2\cdot x_2}&= 3 &&\rightsquigarrow x=\frac{\ln(3)}{2}
	\end{align*}
	\end{multicols}
\end{lsg}
\aufgabe{Schnittpunkte von Funktionen: } In welchen Punkten schneiden sich die Funktionen?
\begin{multicols}{3}
  \begin{enumerate}
    \item $f(x) = x^2$, $g(x) = 2x$
    \item $f(x) = \frac 1 2 x^4$, $g(x) = x^2+4$
  \end{enumerate}
\end{multicols}

\begin{lsg}{}
  \begin{multicols}{2}
    \begin{enumerate}
      \item \begin{align*}
        x^2 &= 2x\\
        x^2-2x &= 0\\
        \intertext{Mit der Mitternachtsformel: }
        x_1 &= 0, x_2 =2\\
        \intertext{Bestimmung der Werte der Stellen: }
        f(0) &= 0, f(2) = 4\\
        \LM &= \left\{\ (0|0),(2|4) \ \right\}
      \end{align*}

      \columnbreak
      \item
      \begin{align*}
        \frac 1 2 x^4 &= x^2+4\\
        \frac 1 2 x^4 -x^2 -4 &= 0\\
      \end{align*}
      Substitution $x^2=u$
      \begin{align*}
        \frac 1 2& u^2 -u -4 = 0\\
        u_{1,2}&=\frac{1\pm\sqrt{1-4*0.5*(-4)}}{1}\\
        &= 1\pm\sqrt{9}\\
        u_1 &= 4\quad u_2=-2\\
      \end{align*}
      Resubstitution von $u_1$ ($u_2 < 0 \rightarrow $ nicht reelles Ergebnis):
      \begin{align*}
        x^2=4\\
        x_1 = -2, x_2 = 2\\
        \intertext{Bestimmung der Werte der Stellen: }
        g(-2)=8, g(2)=8\\
        \LM = \left\{\ (-2|8),(2|8) \ \right\}
      \end{align*}
    \end{enumerate}
  \end{multicols}
\end{lsg}

\aufgabe{Exponentialfunktion}
\begin{enumerate}
	\item Gib $f(x)=25^x$ als natürliche Exponentialfunktion an.
	\item	Wie unterscheidet sich der Graph von $-\e^{-x}$ von $\e^x$? Formuliere die Erklärung schrittweise.
\end{enumerate}
\begin{lsg}{}
	\begin{enumerate}
		\item $\e^{\ln(25)\cdot x}$
		\item 	\begin{itemize}
					\item $-\e^x$ ist zu $ \e^x$ an der x-Achse gespiegelt
					\item $\e^{-x}$ ist zu $\e^x$ an der y-Achse gespiegelt
				\end{itemize}	
				$\Rightarrow -\e^{-x}$ ist zu $\e^x$ an der x- und der y-Achse gespiegelt	 
	\end{enumerate}
\end{lsg}

\aufgabe{Differentialrechnung: }
\begin{enumerate}
	\item Gegeben sei folgende Funktion: $F(x,y)=2x^3-5y+x^2+10x-10$. Bestimmen Sie die Ableitung der durch $F(x,y)=0$ implizit gegebenen Funktion $y=h(x)$.
	\item Gegeben sind die Funktionen:
	$f(x) = (u \circ v)(x)$ und $g(x) = (u* v)(x)$
	Bestimme die Ableitungen von $f$ und $g$ für $u(x)=x^2$ und $v(x)=\sin(2x)$
	\item Bestimme jeweils $f_i'(3)$
	\begin{itemize}
		\item $f_1(x) = (x+5)^2$
		\item $f_2(x) = \frac{1}{(x-5)^2}$
	\end{itemize}
\end{enumerate}
\begin{lsg}{}
	\begin{enumerate}
		\item Aus $F(x,y)=0$ und $y=h(x)$ folgt:\begin{alignat*}{2}
		&2x^3-5y+10x-10=0\quad&&|\ \text{Nach $y=$ umformen}\\
		&y=\frac{2}{5} x^3+\frac{1}{5} x^2+2x-2 &&\Rightarrow h(x)\\
		&h'(x)=\frac{6}{5} x^2+\frac{2}{5}x+2
		\end{alignat*}
		\item \begin{align*}
		f'(x)&=\left(u(v(x))\right)'=\left({\sin(2x)}^2\right)'=2\sin(2x)\cdot \cos(2x) \cdot 2=4\sin(2x)\cdot \cos(2x)\\
		g'(x)&=(u(x)\cdot v(x))'=\left(x^2\cdot \sin(2x)\right)'=2x\cdot \sin(2x)+x^2\cdot 2\cos(2x)
		\end{align*}
		\item \begin{align*}
		f_1'(x)&=2(x+5)&\Rightarrow  &f_1'(3)=16\\
		f_2'(x)&=\frac{-2}{(x-5)^3}&\Rightarrow &f_2'(3)=\frac{1}{4}
		\end{align*}
	\end{enumerate}
\end{lsg}
\aufgabe{Tangenten: }\cas
\begin{enumerate}
	\item Gegeben ist die Funktion g mit $g(x)=2x^2+1$
  \item
\end{enumerate}

\aufgabe{Vermischtes: }
\begin{enumerate}
	\item Beschreiben Sie in Worten, wie sich der Graph von $g(x)=3\sin(3(x+1))-3$ von $\sin(x)$ unterscheidet.
	\item Die Gerade $y=x$ und die $x$-Achse begrenzen zusammen mit den Geraden $x=2$ und $x=u$ mit $u>2$ eine Fläche. Bestimmen Sie einen Wert für $u$ so, dass $f(x)=x-\frac{8}{x^2}$ diese Fläche in zwei inhaltsgleiche Teile zerlegt. \cas
\end{enumerate}
\begin{lsg}{}
  \begin{enumerate}
		\item Amplitude 3, Phasenverschiebung um 1 nach links, Verschiebung um 3 nach unten und Änderung der Periode auf $\frac{2\pi}{3}$
    \item $u\approx 3,12$
  \end{enumerate}
\end{lsg}


\aufgabe{Funktionenscharen: }
\begin{enumerate}
  \item Berechne die Nullstellen der Funktionenscharen in Abhängigkeit von $a\in \R, a\neq 0$:
  \begin{itemize}
    \item $f_a(x)=x^2+2ax+9$
    \item $g_a(x)=5ax+15a$
    \item $h_a(x)=x^3-a^2$
    \item $j_a(x)=(x-3a)(x+6a)$
  \end{itemize}
  \item Gegeben ist die Funktionenschar $f_a$ mit $f_a(x)=\frac{1}{a^2}x^3-\frac{3}{9}x^2-9x+5(a+1)$ mit $a<0$.
  \begin{itemize}
    \item Untersuche die Lage des Maxmimums.
    \item Zeige, dass die Maxima aller Scharkurven auf einer Geraden liegen und gib deren Gleichung an.
  \end{itemize}
\end{enumerate}

\aufgabe{Integral: }
\begin{enumerate}
	\item Welche der Auswahlmöglichkeiten können eingesetzt werden?
	\begin{equation*}
		\int\limits_0^5 3x^2+ \frac 1 5 x  \d x = \Box
	\end{equation*}
  \begin{multicols}{4}
    \begin{itemize}
      \item $\stf{6x + \frac{1}{5}}_0^5$
      \item $\bigg[x^3 + 0,1 x^2\bigg]_0^5$
  		\item $127,5$
      \item $\stf{x^3 + \frac{1}{10} x^2}_1^6$
  	\end{itemize}
  \end{multicols}


	\item Bestimmen Sie $\int\limits_0^{10} \e*x \d x$
  \item Berechnen Sie den Gesamtinhalt der Flächen, die durch die Schaubilder der Funktionen $f$ und $g$ eingeschlossen werden:
  \begin{itemize}
    \item $f(x)=x^2, g(x)=2-x^2$
    \item $f(x)=x^3, g(x)=x^2$
    \item $f(x)=x^3, g(x)=x$
    \item $f(x)=x^3-3x, g(x)=2x^2$
  \end{itemize}
\end{enumerate}
\begin{lsg}{}
  \begin{enumerate}
    \item
    \item
    \item
    \begin{itemize}
      \item $\frac 8 3$
      \item $\frac 1 {12}$
      \item $\frac 1 2$
      \item ?
    \end{itemize}
  \end{enumerate}
\end{lsg}


\aufgabe{Rotationskörper: }
\begin{enumerate}
  \item Die Fläche, welche von der $x$-Achse und dem Graphen der Funktionen vollständig eingeschlossen wird, rotiert um die $x$-Achse.
  Berechne den Rauminhalt des entstandenen Körpers.
  \begin{itemize}
    \item $f(x)=x^2-2x$
    \item $g(x)=\sqrt{x}*(x-2)$
    \item $h(x)=\frac 1 3 x^2-x$
    \item $j(x)=x^2-5x+4$
  \end{itemize}
  \item Die Fläche, welche von den Graphen der Funktionen vollständig eingeschlossen wird, rotiert um die $x$-Achse.
  Berechne den Rauminhalt des entstandenen Körpers.
  \begin{itemize}
    \item $f(x)= -x^2+4, \quad g(x)= x+2$
    \item $h(x)= x^2-x+1, \quad j(x)= 4x-3$
  \end{itemize}
\end{enumerate}


\end{document}
