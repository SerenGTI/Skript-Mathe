\documentclass[a4paper, oneside]{article}
\usepackage[utf8]{inputenc}
\usepackage[ngerman]{babel}
\usepackage[top=2.5cm, bottom=3cm, outer=2.5cm, inner=2.5cm, heightrounded]{geometry}
\usepackage{graphicx}
\usepackage{morefloats}
\usepackage{wrapfig}
\usepackage{hyperref}
\usepackage{cite}
\usepackage{siunitx}
\usepackage[default]{sourcesanspro}
\usepackage[T1]{fontenc}
\usepackage{url}
\usepackage{marginnote}
\usepackage[font=footnotesize]{caption}
\usepackage{color}
\usepackage{xcolor}
\usepackage{multicol}
\usepackage[fleqn]{mathtools}
\usepackage{amssymb}
\usepackage{wrapfig}
\usepackage[noindentafter]{titlesec}
\usepackage{fancyhdr}
\usepackage{lastpage}
\usepackage{comment}

%% LÖSUNGEN ANZEIGEN
\newif\ifshow
%\showtrue
\showfalse

%%%SECTIONING
\renewcommand*{\marginfont}{\noindent\rule{0pt}{0.7\baselineskip}\footnotesize}

\newcommand{\aufgabe}[1]{\subsection{#1}}
\newcommand{\loesung}[1]{\subsubsection{#1}}

\newcommand{\simpleset}[1]{\ensuremath \left\{ #1 \right\}}
\newcommand{\ematrix}[2]{\renewcommand{\arraystretch}{1}\ensuremath\left(\begin{array}{@{}#1@{}}#2\end{array}\right)}

\renewcommand{\theenumi}{\alph{enumi})}
\renewcommand{\labelenumi}{\text{\theenumi}}

\newcounter{aufgabe}
%\newenvironment{lsg}{\loesung}{}
\ifshow
  \newenvironment{lsg}{\loesung}{}
\else
  \excludecomment{lsg}
\fi

\newenvironment{inhalt}
  {\paragraph{Inhalt des Übungsblatts:}\itemize\let\origitem\item}
  {\enditemize\vspace{2em}}

\newcommand{\R}{\ensuremath\mathbb{R}}
\newcommand{\N}{\ensuremath\mathbb{N}}
\newcommand{\Z}{\ensuremath\mathbb{Z}}
\newcommand{\LM}{\ensuremath\mathbb{L}}
\newcommand{\intd}{\ensuremath\mathrm{d}}
\newcommand{\e}{\ensuremath\mathrm{e}}
\renewcommand{\d}{\,\mathrm{d}}
\newcommand{\stf}[1]{\ensuremath \left[ #1 \right]}

\newcommand{\cas}{\hfill (CAS)}
\newcommand{\seite}[1]{\textit{(S. #1)}}

\newcommand{\vektor}[1]{\ensuremath\begin{pmatrix} #1 \end{pmatrix}}


\everymath{\displaystyle}

%Malpunkte
\mathcode`\*="8000
{\catcode`\*\active\gdef*{\cdot}}

%SECTION
\titleformat{\section}
{\clearpage\setcounter{aufgabe}{0}\vspace{1em}\Large\raggedright\bfseries}
{}
{0pt}
{}

\titleformat{\subsection}[runin]
{\stepcounter{aufgabe}\vspace{1px}\normalfont\raggedright\bfseries}
{A\theaufgabe: }
{0pt}
{\ }

\titleformat{\subsubsection}[runin]
{\normalfont\raggedright\bfseries}
{Lösung \theaufgabe: }
{0pt}
{\ }


%FANCYHDR
\pagestyle{fancy}
\lhead{\small Simon König\\ Joshua Fabian}
\rhead{\small Mathecrashkurs 2018}
\cfoot{Seite \thepage\thinspace von\thinspace\pageref{LastPage}}
\lfoot{}
\renewcommand{\headrulewidth}{0.5pt}
\renewcommand{\footrulewidth}{0pt}

\title{Mathe-Crashkurs 2018 - Übungsblatt}
\date{\today}
\author{Simon König, Joshua Fabian}

\chead{\Large Übungsblatt 3}

\usepackage{xfrac}

\begin{document}
\begin{inhalt}
	\item Lagebeziehungen \seite{77}, Abstände \seite{79}
	\item Winkelberechnungen und Spiegelungen \seite{83}
\end{inhalt}


\aufgabe{Abstands- und Lageberechnungen: }
Gegeben sind die Ebene \\$E: \left[ \vec x -  \vektor{-1 \\ 4\\ -3}\right]*\vektor{8\\ 1 \\-4}=0$ und die Gerade $g: \vec x=\vektor{7\\5\\-7}+t*\vektor{1\\-4\\1}$.
\begin{enumerate}
	\item Zeigen Sie, dass $E$ und $g$ parallel zueinander sind.
	\item Bestimmen Sie den Abstand von $E$ und $g$.
\end{enumerate}

\begin{lsg}{}
	\begin{enumerate}
		\item Dafür müssen der Normalenvektor von $E$ und der Richtungsvektor von $g$ orthogonal sein.
		\begin{equation*}
			\vektor{8\\ 1 \\-4}*\vektor{1\\-4\\1}=8-4-4=0
		\end{equation*}
		\item Aufstellen einer Hilfsgeraden:
		\begin{align*}
			h: \vec x=\vektor{7\\5\\-7}+r*\vektor{8\\ 1 \\-4}
		\end{align*}
		Einsetzen von $h$ in $E$:
		\begin{align*}
			&\left[ \left[\vektor{7\\5\\-7}+r*\vektor{8\\ 1 \\-4}\right] -  \vektor{-1 \\ 4\\ -3}\right]*\vektor{8\\ 1 \\-4}=0\\
			&\left[ \vektor{7+8r\\ 5+r \\-7-4r} -  \vektor{-1 \\ 4\\ -3}\right]*\vektor{8\\ 1 \\-4}=0\\
			&\vektor{8+8r\\ 1+r \\-4-4r}*\vektor{8\\ 1 \\-4}=0\\
			&8(8+8r)+(1+r)-4(-4-4r)=0\\
			&64+64r+1+r+16+16r=0\\
			&81+81r=0\\
			&1+r=0\\
			&r=-1\\
			\intertext{Schnittpunkt mit der Ebene durch Einsetzen von $r$ in h:}
			&\vektor{7\\5\\-7}-\vektor{8\\ 1 \\-4}=\vektor{-1 \\ 4 \\ -3}\\
			\intertext{Abstand zwischen den Punkten (Stützpunkt von $g$ und Schnittpunkt $h$ mit $E$):}
			&\vektor{-1 \\ 4 \\ -3} \text{ und } \vektor{7\\5\\-7} \\
			&d(E,g)=\sqrt{8^2+1+(-4)^2}=\sqrt{81}=9
		\end{align*}
	\end{enumerate}
\end{lsg}



\aufgabe{Schnittgerade: }
Gib eine Gleichung der Schnittgeraden der Ebenen $E: x_1-x_2+2x_3=7$ und $F:6x_1+x_2-x_3+7=0$ an.
\begin{lsg}{}
	$E$ und $F$ umformen in Parameterform durch finden von senkrechten Vektoren zum Normalenvektor:
	\begin{align*}
		&v_1-v_2+2v_3=0 \rightarrow\vec v=\vektor{1\\1\\0}
		&u_1-u_2+2u_3=0 \rightarrow\vec u=\vektor{2\\0\\1}\\
		&E: \vec x=\vektor{7\\0\\0}+s*\vektor{1\\1\\0}+t*\vektor{2\\0\\1}
	\end{align*}
	\begin{align*}
		&6v_1+v_2-v_3=0 \rightarrow\vec v=\vektor{1\\0\\6}
		&6u_1+u_2-u_3=0 \rightarrow\vec u=\vektor{0\\1\\1}\\
		&E: \vec x=\vektor{0\\0\\7}+s*\vektor{1\\0\\6}+t*\vektor{0\\1\\1}
	\end{align*}
	Gleichsetzen der Ebenen durch $\vec x$:
	\begin{align*}
		\vektor{0\\0\\7}+a*\vektor{1\\0\\6}+b*\vektor{0\\1\\1}=\vektor{7\\0\\0}+c*\vektor{1\\1\\0}+d*\vektor{2\\0\\1}\\
		a*\vektor{1\\0\\6}+b*\vektor{0\\1\\1}-c*\vektor{1\\1\\0}-d*\vektor{2\\0\\1}=\vektor{7\\0\\-7}
	\end{align*}
	Aufstellen einer geeigneten Matrix:
	\begin{align*}
		\ematrix{rrrr|r}
		{1 & 0 & -1 & -2 & 7\\
		0 & 1 & -1 & 0 & 0\\
		6 & 1 & 0 & -1 & -7}\rightsquigarrow
		\ematrix{rrrr|r}
		{1 & 0 & 0 & \sfrac{-3}{7} & 0\\
		0 & 1 & 0 & \sfrac{11}{7} & -7\\
		0 & 0 & 1 & \sfrac{11}{7} & -7}
	\end{align*}
	Einsetzen in die obere Gleichung:
	\begin{align*}
		&\frac37*\vektor{1\\0\\6}+(-7-\frac{11}{7})*\vektor{0\\1\\1}-(-7-\frac{11}{7})*\vektor{1\\1\\0}-d*\vektor{2\\0\\1}=\vektor{7\\0\\-7}\\
		&\vektor{\sfrac37\\0\\\sfrac{18}{7}}+\vektor{0\\\sfrac{-60}{7}\\\sfrac{-60}{7}}+\vektor{\sfrac{60}{7}\\\sfrac{60}{7}\\0}-d*\vektor{2\\0\\1}=\vektor{7\\0\\-7}\\
		&g: \vec x = \vektor{2\\0\\1}+d*\vektor{2\\0\\1}
	\end{align*}

\end{lsg}

\end{document}
