\documentclass[a4paper, oneside]{article}
\usepackage[utf8]{inputenc}
\usepackage[ngerman]{babel}
\usepackage[top=2.5cm, bottom=3cm, outer=2.5cm, inner=2.5cm, heightrounded]{geometry}
\usepackage{graphicx}
\usepackage{morefloats}
\usepackage{wrapfig}
\usepackage{hyperref}
\usepackage{cite}
\usepackage{siunitx}
\usepackage[default]{sourcesanspro}
\usepackage[T1]{fontenc}
\usepackage{url}
\usepackage{marginnote}
\usepackage[font=footnotesize]{caption}
\usepackage{color}
\usepackage{xcolor}
\usepackage{multicol}
\usepackage[fleqn]{mathtools}
\usepackage{amssymb}
\usepackage{wrapfig}
\usepackage[noindentafter]{titlesec}
\usepackage{fancyhdr}
\usepackage{lastpage}
\usepackage{comment}

%% LÖSUNGEN ANZEIGEN
\newif\ifshow
%\showtrue
\showfalse

%%%SECTIONING
\renewcommand*{\marginfont}{\noindent\rule{0pt}{0.7\baselineskip}\footnotesize}

\newcommand{\aufgabe}[1]{\subsection{#1}}
\newcommand{\loesung}[1]{\subsubsection{#1}}

\newcommand{\simpleset}[1]{\ensuremath \left\{ #1 \right\}}
\newcommand{\ematrix}[2]{\renewcommand{\arraystretch}{1}\ensuremath\left(\begin{array}{@{}#1@{}}#2\end{array}\right)}

\renewcommand{\theenumi}{\alph{enumi})}
\renewcommand{\labelenumi}{\text{\theenumi}}

\newcounter{aufgabe}
%\newenvironment{lsg}{\loesung}{}
\ifshow
  \newenvironment{lsg}{\loesung}{}
\else
  \excludecomment{lsg}
\fi

\newenvironment{inhalt}
  {\paragraph{Inhalt des Übungsblatts:}\itemize\let\origitem\item}
  {\enditemize\vspace{2em}}

\newcommand{\R}{\ensuremath\mathbb{R}}
\newcommand{\N}{\ensuremath\mathbb{N}}
\newcommand{\Z}{\ensuremath\mathbb{Z}}
\newcommand{\LM}{\ensuremath\mathbb{L}}
\newcommand{\intd}{\ensuremath\mathrm{d}}
\newcommand{\e}{\ensuremath\mathrm{e}}
\renewcommand{\d}{\,\mathrm{d}}
\newcommand{\stf}[1]{\ensuremath \left[ #1 \right]}

\newcommand{\cas}{\hfill (CAS)}
\newcommand{\seite}[1]{\textit{(S. #1)}}

\newcommand{\vektor}[1]{\ensuremath\begin{pmatrix} #1 \end{pmatrix}}


\everymath{\displaystyle}

%Malpunkte
\mathcode`\*="8000
{\catcode`\*\active\gdef*{\cdot}}

%SECTION
\titleformat{\section}
{\clearpage\setcounter{aufgabe}{0}\vspace{1em}\Large\raggedright\bfseries}
{}
{0pt}
{}

\titleformat{\subsection}[runin]
{\stepcounter{aufgabe}\vspace{1px}\normalfont\raggedright\bfseries}
{A\theaufgabe: }
{0pt}
{\ }

\titleformat{\subsubsection}[runin]
{\normalfont\raggedright\bfseries}
{Lösung \theaufgabe: }
{0pt}
{\ }


%FANCYHDR
\pagestyle{fancy}
\lhead{\small Simon König\\ Joshua Fabian}
\rhead{\small Mathecrashkurs 2018}
\cfoot{Seite \thepage\thinspace von\thinspace\pageref{LastPage}}
\lfoot{}
\renewcommand{\headrulewidth}{0.5pt}
\renewcommand{\footrulewidth}{0pt}

\title{Mathe-Crashkurs 2018 - Übungsblatt}
\date{\today}
\author{Simon König, Joshua Fabian}

\chead{\Large Übungsblatt 3}

\usepackage{xfrac}

\begin{document}
\begin{inhalt}
	\item Lagebeziehungen \seite{71}, Abstände \seite{73}
	\item Winkelberechnungen und Spiegelungen \seite{77}
\end{inhalt}



\aufgabe{Lagebeziehungen und Ebene aufstellen: } Gegeben sind die Geraden $g$ und $h$:
\begin{equation*}
	g:\vec x=\vektor{2\\-3\\3}+t*\vektor{-1\\-1\\-1} \text{ und } h:\vec x=\vektor{4\\4\\4}+s*\vektor{1\\1\\1}
\end{equation*}
\begin{enumerate}
	\item Welche der beiden Geraden geht durch den Ursprung?
	\item Wie liegen die beiden Geraden zueinander?
	\item Gib eine Gleichung der Ebene an, in der beide Geraden liegen.
\end{enumerate}
\begin{lsg}{}
	\begin{enumerate}
		\item $h$ geht (mit $s=-4$) durch den Ursprung.
		\item Sie sind parallel denn die beiden Richtungsvektoren sind vielfache voneinander.
		\item Stützvektor der Ebene ist der Ursprung, beide Stützvektoren der Geraden bilden die Spannvektoren der Ebene:
		\begin{equation*}
			E: \vec x=t*\vektor{2\\-3\\3}+s*\vektor{4\\4\\4}
		\end{equation*}
	\end{enumerate}
\end{lsg}


\aufgabe{Abstands- und Lageberechnungen: }
Gegeben sind die Ebene $E$ und die Gerade $g$:
\begin{equation*}
	E: \left[ \vec x -  \vektor{-1 \\ 4\\ -3}\right]*\vektor{8\\ 1 \\-4}=0,\quad
	g: \vec x=\vektor{7\\5\\-7}+t*\vektor{1\\-4\\1}
\end{equation*}
\begin{enumerate}
	\item Zeige, dass $E$ und $g$ parallel zueinander sind.
	\item Bestimme den Abstand von $E$ und $g$.
\end{enumerate}

\begin{lsg}{}
	\begin{enumerate}
		\item Dafür müssen der Normalenvektor von $E$ und der Richtungsvektor von $g$ orthogonal sein.
		\begin{equation*}
			\vektor{8\\ 1 \\-4}*\vektor{1\\-4\\1}=8-4-4=0
		\end{equation*}
		\item Aufstellen einer Hilfsgeraden:
		\begin{align*}
			h: \vec x=\vektor{7\\5\\-7}+r*\vektor{8\\ 1 \\-4}
		\end{align*}
		Einsetzen von $h$ in $E$:
		\begin{align*}
			&\left[ \left[\vektor{7\\5\\-7}+r*\vektor{8\\ 1 \\-4}\right] -  \vektor{-1 \\ 4\\ -3}\right]*\vektor{8\\ 1 \\-4}=0\\
			&\left[ \vektor{7+8r\\ 5+r \\-7-4r} -  \vektor{-1 \\ 4\\ -3}\right]*\vektor{8\\ 1 \\-4}=0\\
			&\vektor{8+8r\\ 1+r \\-4-4r}*\vektor{8\\ 1 \\-4}=0\\
			&8(8+8r)+(1+r)-4(-4-4r)=0\\
			&64+64r+1+r+16+16r=0\\
			&81+81r=0\\
			&1+r=0\\
			&r=-1\\
			\intertext{Schnittpunkt mit der Ebene durch Einsetzen von $r$ in h:}
			&\vektor{7\\5\\-7}-\vektor{8\\ 1 \\-4}=\vektor{-1 \\ 4 \\ -3}\\
			\intertext{Abstand zwischen den Punkten (Stützpunkt von $g$ und Schnittpunkt $h$ mit $E$):}
			&\vektor{-1 \\ 4 \\ -3} \text{ und } \vektor{7\\5\\-7} \\
			&d(E,g)=\sqrt{8^2+1+(-4)^2}=\sqrt{81}=9
		\end{align*}
	\end{enumerate}
\end{lsg}


\aufgabe{Abstandsberechnungen: }
\begin{enumerate}
	\item Wie lauten die Koordinaten von $Q$, wenn $P(1|3|-4)$ und $\overrightarrow{PQ}=\vektor{-4\\2\\4}$ sind?
	Wie groß ist der Abstand $d(P,Q)$?
	\item 
\end{enumerate}

\begin{lsg}{}
	\begin{enumerate}
		\item \begin{equation*}
			\overrightarrow{OQ}=\overrightarrow{OP}+\overrightarrow{PQ}\rightsquigarrow Q(-3|5|0)
	\end{equation*}
	Für den Abstand gilt: $d(P,Q)=\sqrt{(-4)^2+2^2+4^2}=6$
	\end{enumerate}
\end{lsg}



\aufgabe{Lage- und Vektorrechnung: } Gegeben sind die Punkte $A(2|-1|2), B(5|-2|4), C(4|3|1)$ und $D(3|8|3)$. \cas
\begin{enumerate}
	\item Zeige, dass die Punkte $A,B,C$ und $D$ nicht in einer Ebene liegen.
	\item Prüfe, ob sich die Geraden $g$ durch $A$ und $B$ sowie die Gerade $h$ durch $A$ und $C$ orthogonal schneiden.
	\item Untersuche, ob das Dreieck $ABC$ gleichschenklig ist.
\end{enumerate}
\begin{lsg}{}
	\begin{enumerate}
		\item Aufstellen einer Ebenengleichung, die $B,C$ und $D$ enthält:
		\begin{equation*}
			E: \vec x=\vektor{5\\-2\\4}+r*\vektor{-1\\5\\-3}+s*\vektor{-2\\10\\-1}
		\end{equation*}
		Einsetzen von $A$ in $\vec x$, lösen des LGS mit dem CAS: Keine Lösung, d.h. liegen nicht in einer Ebene.
		\item\begin{align*}
			\overrightarrow{AB}*\overrightarrow{AC}&\overset!=0\\
			\vektor{3\\-1\\2}*\vektor{2\\4\\-1}&=3*2-4-2=0
		\end{align*}
		Die Geraden schneiden sich also rechtwinklig.
		\item Da $ABC$ ein rechtwinkliges Dreieck ist, müssen nur die Katheten, also die Längen $|\overrightarrow{AB}|$ und $|\overrightarrow{AC}|$ betrachtet werden.\begin{align*}
			&|\overrightarrow{AB}|=\sqrt{9+1+4}=\sqrt{14}\\
			&|\overrightarrow{AC}|=\sqrt{4+16+1}=\sqrt{21}
		\end{align*}
		Da $\sqrt{21}\neq \sqrt{14}$ ist, kann das Dreieck nicht gleichschenklig sein.
	\end{enumerate}
\end{lsg}




\aufgabe{Geometriegewurschtel: } Die Grundfläche einer vierseitigen Pyramide liegt in der Ebene $E$ und hat die Eckpunkte $A(0|1|1),B(2|4|-5),C(-1|10|-3),D(-3|7|3)$.
Die Spitze der Pyramide liegt auf der Geraden
\begin{equation*}
	g:\vec x=\vektor{9\\10\\12}+s*\vektor{1\\3\\17}
\end{equation*}
\begin{enumerate}
	\item Zeige, dass die Grundfläche der Pyramide ein Quadrat ist. Bestimmen Sie eine Koordinatengleichung der Ebene $E$.
	\item Die Pyramidenspitze kann auf der Geraden $g$ so gewählt werden, dass die vier von der Grundfläche zur Pyramidenspitze verlaufenden Kanten gleich lang sind. Berechne die Koordinaten der Spitze für diesen Fall.
	\item Die Pyramidenspitze $S$ kann auf der Geraden $g$ auch so gewählt werden, dass die Seitenfläche $ABS$ orthogonal ist. Berechnen Sie die Koordinaten der Spitze für diesen Fall.
\end{enumerate}
\begin{lsg}{}
	\begin{enumerate}
		\item Es müssen die \textbf{vier} Seitenlängen auf Gleichheit geprüft werden. Zusätzlich muss mindestens ein Winkel auf Rechtwinkligkeit geprüft werden. (Alle Seitenlängen sind $7$)

		Koordinatengleichung:
		\begin{align*}
			&E:\vektor{2\\4\\-5}+t*\vektor{2\\3\\-6}+r*\vektor{-3\\6\\2}\\
			\rightsquigarrow&E:\left[\vec x-\vektor{2\\4\\-5}\right]*\vektor{2\\3\\-6}\times\vektor{-3\\6\\2}=0\\
			\Leftrightarrow &E:\left[\vec x-\vektor{2\\4\\-5}\right]*\vektor{42\\14\\21}=0\\
			\rightsquigarrow&E:42x_1+14x_2+21x_3=35
		\end{align*}
		\item Damit alle vier Kanten die gleiche Länge haben, muss sich die Spitze rechtwinklig über dem Schnittpunkt der Diagonalen, also dem Mittelpunkt $M$ befinden.
		\begin{equation*}
			\overrightarrow{OM}=\overrightarrow{OA}+\frac12\overrightarrow{AC}=\vektor{0\\1\\1}+\frac12\vektor{-1\\9\\-4}
			=\vektor{-0,5\\5,5\\-1}
		\end{equation*}
		Aufstellen einer Gerade, die rechtwinklig zur Grundebene $E$ verläuft:
		\begin{equation*}
			m:\vec x=\vektor{-0,5\\5,5\\-1}+q*\vektor{42\\14\\21}
		\end{equation*}
		Die gesuchte Spitze ist der Schnittpunkt von $g$ mit $m$:
		Aufstellen eines LGS und Lösen mit dem CAS liefert die Ergebnise $q=\frac 3{14}, s=-0,5$
		Einsetzen von $s$:
		\begin{equation*}
			\overrightarrow{OS}=\vektor{9\\10\\12}-0,5*\vektor{1\\3\\17}=\vektor{\sfrac{17}2\\\sfrac{17}2\\\sfrac{7}2}
		\end{equation*}
		\item Aufstellen einer Ebene $K$, die rechtwinklig auf $E$ steht und die Punkte $A$ und $B$ enthält.
		\begin{equation*}
			K: \vec x=\vektor{0\\1\\1}+r*\vektor{2\\3\\-6}+s*\vektor{42\\14\\21}
		\end{equation*}
		Gesucht ist der Schnittpunkt von $K$ und $g$.
		\begin{align*}
			\underbrace{\vektor{9\\10\\12}+a*\vektor{1\\3\\17}}_g&=\underbrace{\vektor{0\\1\\1}+b*\vektor{2\\3\\-6}+c*\vektor{42\\14\\21}}_K\\
			\Leftrightarrow a*\vektor{1\\3\\17}-b*\vektor{2\\3\\-6}-c*\vektor{42\\14\\21}&=\vektor{-9\\-9\\-11}\\
			\ematrix{rrr|r}{1&-2&-42&-9\\3&-3&-14&-9\\17&6&-21&-11}&\rightsquigarrow\ematrix{rrr|r}{1&0&0&-1\\0&1&0&\sfrac{10}7\\0&0&1&\sfrac6{49}}
		\end{align*}
		Einsetzen von $a=s=-1$ in die Geradengleichung ergibt den Ortsvektor der Spitze:
		\begin{equation*}
			\overrightarrow{OS}=\vektor{9\\10\\12}-\vektor{1\\3\\17}=\vektor{8\\7\\-5}
		\end{equation*}
	\end{enumerate}
\end{lsg}

\end{document}
