\input{master/layout}
\chead{\Large Joshua Aufgabensammlung}


\begin{document}
\aufgabe{Gleichung lösen}
Löse die Gleichung $\e^{5x}-\e^{3x}=6\e^x$.\\
\textit{Hinweis: Du brauchst ungefähr alle gelernten Methoden!}
\begin{lsg}{}
	\begin{multicols}{2}
	\begin{align*}
		\e^{5x}-\e^{3x}&=6\e^x &&|-6\e^x\\
		\e^{5x}-\e^{3x}-6\e^x&=0 &&|\ \text{Ausklammern}\\
		\e^x\cdot(\e^{4x}-\e^{2x}-6)&=0 &&|\ \text{Nullprodukt und Substitution: } z=\e^{2x}\\
		z^2-z-6&=0 &&|\ \text{Mitternachtsformel}\\
		z_1=-2 \quad z_2&=3&&|\ \text{Resubstitution: } z_{1/2}=\e^{2\cdot x_{1/2}}\\
		\e^{2\cdot x_1}&=-2 &&\Rightarrow \text{nicht möglich ($\e^x$ immer $>0$)}\\
		\e^{2\cdot x_2}&= 3 &&\rightsquigarrow x=\frac{\ln(3)}{2}
	\end{align*}
	\end{multicols}
\end{lsg}
\aufgabe{Schnittpunkte von Funktionen: } In welchen Punkten schneiden sich die Funktionen?
\begin{multicols}{3}
  \begin{enumerate}
    \item $f(x) = x^2$, $g(x) = 2x$
    \item $f(x) = \frac 1 2 x^4$, $g(x) = x^2+4$
  \end{enumerate}
\end{multicols}

\begin{lsg}{}
  \begin{multicols}{2}
    \begin{enumerate}
      \item \begin{align*}
        x^2 &= 2x\\
        x^2-2x &= 0\\
        \intertext{Mit der Mitternachtsformel: }
        x_1 &= 0, x_2 =2\\
        \intertext{Bestimmung der Werte der Stellen: }
        f(0) &= 0, f(2) = 4\\
        \LM &= \left\{\ (0|0),(2|4) \ \right\}
      \end{align*}

      \columnbreak
      \item
      \begin{align*}
        \frac 1 2 x^4 &= x^2+4\\
        \frac 1 2 x^4 -x^2 -4 &= 0\\
      \end{align*}
      Substitution $x^2=u$
      \begin{align*}
        \frac 1 2& u^2 -u -4 = 0\\
        u_{1,2}&=\frac{1\pm\sqrt{1-4*0.5*(-4)}}{1}\\
        &= 1\pm\sqrt{9}\\
        u_1 &= 4\quad u_2=-2\\
      \end{align*}
      Resubstitution von $u_1$ ($u_2 < 0 \rightarrow $ nicht reelles Ergebnis):
      \begin{align*}
        x^2=4\\
        x_1 = -2, x_2 = 2\\
        \intertext{Bestimmung der Werte der Stellen: }
        g(-2)=8, g(2)=8\\
        \LM = \left\{\ (-2|8),(2|8) \ \right\}
      \end{align*}
    \end{enumerate}
  \end{multicols}
\end{lsg}

\aufgabe{Exponentialfunktion}
\begin{enumerate}
	\item Gib $f(x)=25^x$ als natürliche Exponentialfunktion an.
	\item	Wie unterscheidet sich der Graph von $-\e^{-x}$ von $\e^x$? Formuliere die Erklärung schrittweise.
\end{enumerate}
\begin{lsg}{}
	\begin{enumerate}
		\item $\e^{\ln(25)\cdot x}$
		\item 	\begin{itemize}
					\item $-\e^x$ ist zu $ \e^x$ an der x-Achse gespiegelt
					\item $\e^{-x}$ ist zu $\e^x$ an der y-Achse gespiegelt
				\end{itemize}
				$\Rightarrow -\e^{-x}$ ist zu $\e^x$ an der x- und der y-Achse gespiegelt
	\end{enumerate}
\end{lsg}

\aufgabe{Differentialrechnung: }
\begin{enumerate}
	\item Gegeben sei folgende Funktion: $F(x,y)=2x^3-5y+x^2+10x-10$. Bestimmen Sie die Ableitung der durch $F(x,y)=0$ implizit gegebenen Funktion $y=h(x)$.
	\item Gegeben sind die Funktionen:
	$f(x) = (u \circ v)(x)$ und $g(x) = (u* v)(x)$
	Bestimme die Ableitungen von $f$ und $g$ für $u(x)=x^2$ und $v(x)=\sin(2x)$
	\item Bestimme jeweils $f_i'(3)$
	\begin{itemize}
		\item $f_1(x) = (x+5)^2$
		\item $f_2(x) = \frac{1}{(x-5)^2}$
	\end{itemize}
\end{enumerate}
\begin{lsg}{}
	\begin{enumerate}
		\item Aus $F(x,y)=0$ und $y=h(x)$ folgt:\begin{alignat*}{2}
		&2x^3-5y+10x-10=0\quad&&|\ \text{Nach $y=$ umformen}\\
		&y=\frac{2}{5} x^3+\frac{1}{5} x^2+2x-2 &&\Rightarrow h(x)\\
		&h'(x)=\frac{6}{5} x^2+\frac{2}{5}x+2
		\end{alignat*}
		\item \begin{align*}
		f'(x)&=\left(u(v(x))\right)'=\left({\sin(2x)}^2\right)'=2\sin(2x)\cdot \cos(2x) \cdot 2=4\sin(2x)\cdot \cos(2x)\\
		g'(x)&=(u(x)\cdot v(x))'=\left(x^2\cdot \sin(2x)\right)'=2x\cdot \sin(2x)+x^2\cdot 2\cos(2x)
		\end{align*}
		\item \begin{align*}
		f_1'(x)&=2(x+5)&\Rightarrow  &f_1'(3)=16\\
		f_2'(x)&=\frac{-2}{(x-5)^3}&\Rightarrow &f_2'(3)=\frac{1}{4}
		\end{align*}
	\end{enumerate}
\end{lsg}
\aufgabe{Tangenten: }\cas
\begin{enumerate}
	\item Gegeben ist die Funktion $g$ mit $\ g(x)=2x^2+1$. Bestimme die Gleichung der Tangente und der Normalen an den Graphen G von $g$ im Punkt $P(1|3)$.
  \item Bestimme die Gleichung der Tangente und Normalen an den Graphen G von $g$ mit der Steigung $1$.
  \item Bestimme die Gleichungen der beiden Tangenten an den Graphen G von $g$, die durch den Punkt $Q(3|17)$ gehen und gib die Berührpunkte der Tangenten mit dem Graphen G von $g$ an.
\end{enumerate}
\begin{lsg}{}
	\begin{align*}
		\text{Tangentengleichung: }  t(x,x_0)&=g'(x_0)*(x-x_0)+g(x_0)\\ 
	\text{Normalengleichung: }n(x,x_0)&=-\frac{1}{g'(x_0)}*(x-x_0)+g(x_0) 
	\end{align*}
	
	\begin{enumerate}
		\item Aus Punkt P(1/3) folgt: $x_0=1,\ g(x_0)=3$ und ableiten ergibt $\ g'(x)=4x$
		\begin{align*}
		\text{Damit gilt: }\ t(x)&=4*(x-1)+3 \\
		\text{und: }\ n(x)&=-\frac{1}{4}*(x-1)+3 
		\end{align*}
		\item \begin{align*}
			\text{Tangentengleichung: }\ g'(x_0)&=1 \rightsquigarrow x_0=\frac{1}{4}\\
			g\left(\frac{1}{4}\right)&=\frac{9}{8}\\
			t(x)&=1*\left(x-\frac{1}{4}\right)+\frac{9}{8}\\
			\text{Normalengleichung:}\ -\frac{1}{g'(x_0)}&=1 \rightsquigarrow x_0=-\frac{1}{4}\\
			g\left(-\frac{1}{4}\right)&=\frac{9}{8}\\
			n(x)&=1*\left(x+\frac{1}{4}\right)+\frac{9}{8}
			\end{align*}
			\item Aus dem Punkt $Q(3|517)$ folgt: $\ t(3)=17$, daher gilt für die Tangentengleichungen:\begin{align*}
			17&=g'(x_0)*(3-x_0)+g(x_0) \rightsquigarrow x_{0,1}=2,\ x_{0,2}=4\\
			t_1(x)&=8*(x-2)+9\\
			t_2(x)&=16*(x-4)+33
			\end{align*}
	\end{enumerate}
\end{lsg}
\aufgabe{Vermischtes: }
\begin{enumerate}
	\item Beschreiben Sie in Worten, wie sich der Graph von $g(x)=3\sin(3(x+1))-3$ von $\sin(x)$ unterscheidet.
	\item Die Gerade $y=x$ und die $x$-Achse begrenzen zusammen mit den Geraden $x=2$ und $x=u$ mit $u>2$ eine Fläche. Bestimmen Sie einen Wert für $u$ so, dass $f(x)=x-\frac{8}{x^2}$ diese Fläche in zwei inhaltsgleiche Teile zerlegt. \cas
\end{enumerate}
\begin{lsg}{}
  \begin{enumerate}
		\item Amplitude 3, Phasenverschiebung um 1 nach links, Verschiebung um 3 nach unten und Änderung der Periode auf $\frac{2\pi}{3}$
    \item Bestimme die Flächeninhalte der Teilflächen:\begin{align*}
    A_1&=\int_2^u\left(x-\left(x-\frac{8}{x^2}\right)\right)\d x&&|\ \text{Fläche zwischen $y=x$ und $f(x)=x-\frac{8}{x^2}$}\\
    A_2&= \int_2^u\left(x-\frac{8}{x^2}\right)\d x&&|\ \text{Fläche zwischen $f(x)=x-\frac{8}{x^2}$ und $x$-Achse}
    \end{align*}
    Gleichsetzten und mittels CAS nach $u$ auflösen ergibt: $u\approx 3,12$
  \end{enumerate}
\end{lsg}


\aufgabe{Funktionenscharen: }
\begin{enumerate}
  \item Berechne die Nullstellen der Funktionenscharen in Abhängigkeit von $a\in \R, a\neq 0$:
  \begin{itemize}
    \item $f_a(x)=x^2+2ax+9$
    \item $g_a(x)=5ax+15a$
    \item $h_a(x)=x^3-a^2$
    \item $j_a(x)=(x-3a)(x+6a)$
  \end{itemize}
  \item Gegeben ist die Funktionenschar $f_a$ mit $f_a(x)=\frac{1}{a^2}x^3-\frac{3}{9}x^2-9x+5(a+1)$ mit $a<0$.
  \begin{itemize}
    \item Untersuche die Lage des Maxmimums.
    \item Gib die Gleichung der Gerade an, auf der die Maxima aller Scharkurven liegen.
  \end{itemize}
\end{enumerate}
\begin{lsg}{}
	\begin{enumerate}
		\item \begin{itemize}
			\item $0=x^2+2ax+9 \rightsquigarrow x_1=\sqrt{a^2-9}-a,\ x_2=-\sqrt{a^2-9}-a$\\
			\item $0=5ax+15a \rightsquigarrow x=-3$\\
			\item $0=x^3-a^2 \rightsquigarrow x=\sqrt[3]{a^2}$\\			
			\item $\text{Nullprodukt:}\ x_1=3a,\ x_2=-6a$
		\end{itemize}
		\item nvm
		
	\end{enumerate}
\end{lsg}
\aufgabe{Integral: }
\begin{enumerate}
	\item Welche der Auswahlmöglichkeiten können eingesetzt werden?
	\begin{equation*}
		\int\limits_0^5 \left(3x^2+ \frac 1 5 x\right)  \d x = \Box
	\end{equation*}
  \begin{multicols}{4}
    \begin{itemize}
      \item $\stf{6x + \frac{1}{5}}_0^5$
      \item $\bigg[x^3 + 0,1 x^2\bigg]_0^5$
  		\item $127,5$
      \item $\stf{x^3 + \frac{1}{10} x^2}_1^6$
  	\end{itemize}
  \end{multicols}


	\item Bestimmen Sie $\int\limits_0^{\ln(10)} \e^x \d x$
  \item Berechnen Sie den Gesamtinhalt der Flächen, die durch die Schaubilder der Funktionen $f$ und $g$ eingeschlossen werden:
  \begin{itemize}
    \item $f(x)=x^2, g(x)=2-x^2$
    \item $f(x)=x^3, g(x)=x^2$
    \item $f(x)=x^3, g(x)=x$
    \item $f(x)=x^3-3x, g(x)=2x^2$
  \end{itemize}
\end{enumerate}
\begin{lsg}{}
  \begin{enumerate}
    \item $\bigg[x^3 + 0,1 x^2\bigg]_0^5$ und $127,5$
    \item
    \item
    \begin{itemize}
      \item $\frac 8 3$
      \item $\frac 1 {12}$
      \item $\frac 1 2$
      \item ?
    \end{itemize}
  \end{enumerate}
\end{lsg}


\aufgabe{Rotationskörper: }
\begin{enumerate}
  \item Die Fläche, welche von der $x$-Achse und dem Graphen der Funktionen vollständig eingeschlossen wird, rotiert um die $x$-Achse.
  Berechne den Rauminhalt des entstandenen Körpers.
  \begin{itemize}
    \item $f(x)=x^2-2x$
    \item $g(x)=\sqrt{x}*(x-2)$
    \item $h(x)=\frac 1 3 x^2-x$
    \item $j(x)=x^2-5x+4$
  \end{itemize}
  \item Die Fläche, welche von den Graphen der Funktionen vollständig eingeschlossen wird, rotiert um die $x$-Achse.
  Berechne den Rauminhalt des entstandenen Körpers.
  \begin{itemize}
    \item $f(x)= -x^2+4, \quad g(x)= x+2$
    \item $h(x)= x^2-x+1, \quad j(x)= 4x-3$
  \end{itemize}
\end{enumerate}

\aufgabe{Lineare Gleichungssysteme}
Löse das Gleichungssystem:
\begin{alignat*}{4}
	-5x_1& +x_2& -x_3& = 7\\
	5x_1&  -3x_2& -2_3& = -11\\
	x_1& & x_3& =-1
\end{alignat*}
Interpretiere das LGS und die Lösungsmenge geometrisch.

\aufgabe{Ebenengleichungen: }
Bestimme die Ebene in der angegebenen Darstellungsform:
\begin{enumerate}
	\item $E$ enthält die Punkte $A(2|2|2), B(4|1|3)$ und $C(8|4|5)$. Gib $E$ in Normalenform an. %Parameterform -> Normalenform,
	\item Die gesuchte Ebene $F$ ist die Spiegelebene zwischen $A(1|4|7)$ und $A'(3|2|3)$. Gib die $F$ in Parameterform an. %Normalenform -> Parameterform
	\item Die Ebene $G$ enthält die Gerade $\vec x = \vektor{3\\1\\2}+s*\vektor{2\\0\\-1}$ und ist orthogonal zur Ebene $H:-x_1+x_2+2x_3+2=0$. Gib die Ebene $G$ in Koordinatenform an. %Parameterform -> Koordinatenform
\end{enumerate}

\aufgabe{Schnittgerade: }
Gib eine Gleichung der Schnittgeraden der Ebenen $E: x_1-x_2+2x_3=7$ und $F:6x_1+x_2-x_3+7=0$ an.

\aufgabe{Abstandsberechnungen Teil 1: }
Berechne den Abstand des Punktes $R(6|9|4)$ von der Ebene $E: \left[ \vec x -  \vektor{7 \\ 5\\ 2}\right]*\vektor{2\\ 2 \\1}=0$

\aufgabe{Abstands- und Lageberechnungen: }
Gegeben sind die Ebene \\$E: \left[ \vec x -  \vektor{-1 \\ 4\\ -3}\right]*\vektor{8\\ 1 \\-4}=0$ und die Gerade $g: \vec x=\vektor{7\\5\\-7}+t*\vektor{1\\-4\\1}$.
\begin{enumerate}
	\item Zeigen Sie, dass $E$ und $g$ parallel zueinander sind.
	\item Bestimmen Sie den Abstand von $E$ und $g$.
\end{enumerate}

\aufgabe{Winkelberechnung}
\begin{enumerate}
	\item Berechnen Sie die Schnittwinkel der beiden Geraden $g_i$ und $h_i$:
	\begin{itemize}
		\item $g_1: \vec x = \vektor{2\\2\\-3} + r*\vektor{2\\1\\-1}$ und $h_1: \vec x = \vektor{3\\0\\-1} + s* \vektor{1\\-2\\2}$
		\item
	\end{itemize}
	\item
\end{enumerate}

\aufgabe{Lageberechnungen}
\begin{enumerate}
	\item enum
	\item	enum
\end{enumerate}

\aufgabe{Abstandsberechnung}
\begin{enumerate}
	\item enum
	\item	enum
\end{enumerate}


\aufgabe{Winkelberechnung}
\begin{enumerate}
	\item enum
	\item	enum
\end{enumerate}

\end{document}
