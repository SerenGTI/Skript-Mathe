\input{master/layout}
\chead{\Large Joshua Aufgabensammlung}


\begin{document}



\aufgabe{Winkelberechnung}
\begin{enumerate}
	\item Berechnen Sie die Schnittwinkel der beiden Geraden $g_i$ und $h_i$:
	\begin{itemize}
		\item $g_1: \vec x = \vektor{2\\2\\-3} + r*\vektor{2\\1\\-1}$ und $h_1: \vec x = \vektor{3\\0\\-1} + s* \vektor{1\\-2\\2}$
		\item
	\end{itemize}
	\item
\end{enumerate}

\aufgabe{Lageberechnungen}
\begin{enumerate}
	\item enum
	\item	enum
\end{enumerate}







% IN ÜBUNGSBLATT 4 ANS ENDE?
\aufgabe{Graphanalyse: } (vgl. Abitur 2015)
\begin{multicols}{2}
	Die Abbildung zeigt den Graphen der Ableitungsfunktion $f'$ einer ganzrationalen Funktion $f$.
	Entscheide ob die folgenden Aussagen wahr oder falsch sind. Begründe jeweils Deine Antwort.
	\begin{enumerate}
		\item Der Graph von $f$ hat bei $x=-3$ einen Tiefpunkt.
		\item $f(-2)<f(-1)$
		\item $f''(-2)+f'(-2)<1$
		\item Der Grad der Funktion $f$ ist mindestens vier.
	\end{enumerate}
	\columnbreak

	\centering
	\includegraphics[width=0.7\linewidth]{Graphanalyse.png}
\end{multicols}

\begin{lsg}{}
	\begin{enumerate}
		\item Wahr, Vorzeichenwechsel bei $x=-3$
		\item Wahr, streng monoton steigend im Intervall $[-2;-1]$
		\item Falsch, $f''(-2)+f'(-2)=0+2>1$
		\item Wahr, $f'$ besitzt zwei Extrempunkte $\rightsquigarrow f''$ ist mindestens vom Grad 2. Der Graph der Abbildung könnte auch drei Nullstellen haben, $f'$ ist also mindestens vom Grad 3.
	\end{enumerate}
\end{lsg}



\aufgabe{Extrempunkte}

\begin{lsg}{}

\end{lsg}


\end{document}
