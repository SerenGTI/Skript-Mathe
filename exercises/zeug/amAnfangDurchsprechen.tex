\documentclass[a4paper, oneside]{article}
\usepackage[utf8]{inputenc}
\usepackage[ngerman]{babel}
\usepackage[paperheight=13.5cm,paperwidth=24cm,top=2.5cm,left=2cm,right=2cm,bottom=2cm,heightrounded]{geometry}
\usepackage{graphicx}
\usepackage{wrapfig}
\usepackage{hyperref}
\usepackage{cite}
\usepackage[default]{sourcesanspro}
\usepackage[T1]{fontenc}
\usepackage[font=footnotesize]{caption}
\usepackage{xcolor}
\usepackage{multicol}
\usepackage[fleqn]{mathtools}
\usepackage{amssymb}
\usepackage[noindentafter]{titlesec}
\usepackage{fancyhdr}
\usepackage{lastpage}
\usepackage{wasysym}


\usepackage[framemethod=TikZ]{mdframed}
\mdfdefinestyle{badExpl}{%
  linecolor=red,
  outerlinewidth=1pt,
  roundcorner=4pt,
  innertopmargin=\baselineskip,
  innerbottommargin=\baselineskip,
  innerrightmargin=20pt,
  innerleftmargin=20pt,
  backgroundcolor=gray!10!white
}
\mdfdefinestyle{goodExpl}{%
  linecolor=green,
  outerlinewidth=1pt,
  roundcorner=4pt,
  innertopmargin=\baselineskip,
  innerbottommargin=\baselineskip,
  innerrightmargin=20pt,
  innerleftmargin=20pt,
  backgroundcolor=gray!10!white
}


%%%SECTIONING

\everymath{\displaystyle}

%Malpunkte
\mathcode`\*="8000
{\catcode`\*\active\gdef*{\cdot}}

%SECTION
\titleformat{\section}
{\vspace{1em}\Large\raggedright\bfseries}
{}
{0pt}
{}

\titleformat{\subsection}[runin]
{\vspace{1px}\normalfont\raggedright\bfseries}
{A\theaufgabe: }
{0pt}
{\ }

\titleformat{\subsubsection}[runin]
{\normalfont\raggedright\bfseries}
{Lösung \theaufgabe: }
{0pt}
{\ }


%FANCYHDR
\pagestyle{fancy}
\lhead{\small Simon König \& Joshua Fabian}
\rhead{\small Mathecrashkurs 2018}
\cfoot{}
\lfoot{}
\renewcommand{\headrulewidth}{0.5pt}
\renewcommand{\footrulewidth}{0pt}

\title{Mathe-Crashkurs 2018}
\date{\today}
\author{Simon König, Joshua Fabian}
\chead{}


\setlength\columnsep{50pt}


\begin{document}
\section*{Organisatorisches}
	\begin{itemize}
		\item Wer zum Rauchen raus geht, bitte Tür schließen und weg vom Eingang gehen.
		\item Cafe Schwarz bleibt aus Hygienegründen geschlossen.
		\item Freiweilliges Arbeiten, insbesondere keine Anwesenheitspflicht. Stört die anderen aber nicht.
	\end{itemize}

\clearpage
\section*{Zeug zum Kurs}
\begin{multicols}{2}
	\begin{itemize}
		\setlength\itemsep{3em}
			\item Fragen stellen, unterbrechen bis alles klar ist sonst springen wir durch die Themen und haben das Skript heute Nachmittag fertig.
			\item Überlegt euch was eure Schwachstellen sind, schaut euch auch eure eigenen Klausuren an:\\
			Was lief schief? Was habt ihr noch nicht so richtig verstanden und arbeitet insbesondere daran.

			Übt nicht das was ihr schon könnt, sondern das was nicht so gut läuft.
			\item Es gibt jeden Tag ein Übungsblatt. Ihr dürft logischerweise auch eigene Sachen rechnen.
			\item Die Lösungen zu den Übungsblättern gibt's auch digital als pdf - wichtig dafür: die Lösungen sollten (hoffentlich) richtig sein, sie sind jedoch \emph{nicht} so ausführlich wie sie auf Abiturniveau sein könnten/ sollten! Fragt also bei Unklarheiten zum Lösungsweg oä. nach.
	\end{itemize}
\end{multicols}

\clearpage
\section*{Tipps und Tricks im Abi}
\begin{itemize}
		\item Verwendet Skizzen o.ä. um euch die Aufgabe bildlich darzustellen. Das vereinfacht dann auch dem Korrektor euren Rechenweg nachzuvollziehen.
		\item Aufschrieb kommentieren, der Korrektor soll nicht überlegen müssen was ihr tut! Solange der Korrektor glücklich ist, gibt's auch für falsche Ergebnisse Teilpunkte, also übt das ein bisschen:
	\begin{multicols}{2}
		\begin{mdframed}[style=badExpl]
			\textbf{Nicht so gut:}
			\begin{align*}
				5x+4=3x-2\\
				\Rightarrow x=-3
		 \end{align*}
		\end{mdframed}
		\columnbreak
		\begin{mdframed}[style=goodExpl]
			\textbf{Sehr gut:}\\
			Gleichsetzen der Funktionen:\begin{align*}
				&5x+4=3x-2 \!{\color{red} \checkmark}
				\intertext{Der CAS liefert die Schnittstelle:}
				\overset{\text{CAS}}\Rightarrow& x=-3 \!{\color{red} \checkmark} &\text{\color{red} \smiley{}}
		 \end{align*}
		\end{mdframed}
	\end{multicols}
\end{itemize}





\clearpage
\section*{Grober Zeitplan für die Woche:}
		\paragraph{Montag}
		Themen des Übungsblatts \begin{itemize}
			\item Gleichungen
			\item Monotonie, Achsenschnittpunkte, Trigonometrie
			\item Ableiten
			\item Tangenten, Normalen
		\end{itemize}

		\paragraph{Dienstag}
		Themen des Übungsblatts \begin{itemize}
			\item Extremstellen und -Punkte
			\item Exponentialfunktion
			\item Funktionenscharen
			\item Integral, Rotationskörper, Flächeninhalte
			\item Funktionsanalyse, gebrochenrationale Funktionen, Asymptoten
		\end{itemize}

		\paragraph{Mittwoch}
		Themen des Übungsblatts \begin{itemize}
			\item Wachstum
			\item LGS-Rechnung, Vektorrechnung
			\item Geraden und Ebenen
		\end{itemize}

		\paragraph{Donnerstag}
		Themen des Übungsblatts \begin{itemize}
			\item Lagebeziehungen, Abstände
			\item Winkelberechnungen und Spiegelungen
		\end{itemize}
		Freiwillig Nachmittags ein Probeabitur durchrechnen, wird (hoffentlich) auf Freitag korrigiert und dann durchgesprochen\\
		Alternativ Übungsblatt rechnen (je nachdem wie viel Zeit ist auch beides)\\

\clearpage
		\paragraph{Freitag}
		Themen des Übungsblatts \begin{itemize}
			\item Zufallsexperimente
			\item Kombinatorik, Binomalverteilung und Bernoulli-Versuch
			\item Hypothesentest
			\item Vermischtes
		\end{itemize}
		 Besprechung des Abiturs\\


\clearpage
\begin{align*}
	&f(x)=x^5+2x-1\\
	&g(x)=e^{2x}(x^4+2x)\\
	&h(x)=4\sin\left(\frac 1 8x^2+x\right)+2\\
	&i(x)=\frac{4x^2+2}{e^x}\\
	&j(x)=x^x\\
	&k(x)=
	&l(x)=
	&m(x)=
	&n(x)=
\end{align*}
\end{document}
