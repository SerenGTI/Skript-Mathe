\documentclass[a4paper, oneside]{article}
\usepackage[utf8]{inputenc}
\usepackage[ngerman]{babel}
\usepackage[top=2.5cm, bottom=3cm, outer=2.5cm, inner=2.5cm, heightrounded]{geometry}
\usepackage{graphicx}
\usepackage{morefloats}
\usepackage{wrapfig}
\usepackage{hyperref}
\usepackage{cite}
\usepackage{siunitx}
\usepackage[default]{sourcesanspro}
\usepackage[T1]{fontenc}
\usepackage{url}
\usepackage{marginnote}
\usepackage[font=footnotesize]{caption}
\usepackage{color}
\usepackage{xcolor}
\usepackage{multicol}
\usepackage[fleqn]{mathtools}
\usepackage{amssymb}
\usepackage{wrapfig}
\usepackage[noindentafter]{titlesec}
\usepackage{fancyhdr}
\usepackage{lastpage}
\usepackage{comment}

%% LÖSUNGEN ANZEIGEN
\newif\ifshow
%\showtrue
\showfalse

%%%SECTIONING
\renewcommand*{\marginfont}{\noindent\rule{0pt}{0.7\baselineskip}\footnotesize}

\newcommand{\aufgabe}[1]{\subsection{#1}}
\newcommand{\loesung}[1]{\subsubsection{#1}}

\newcommand{\simpleset}[1]{\ensuremath \left\{ #1 \right\}}
\newcommand{\ematrix}[2]{\renewcommand{\arraystretch}{1}\ensuremath\left(\begin{array}{@{}#1@{}}#2\end{array}\right)}

\renewcommand{\theenumi}{\alph{enumi})}
\renewcommand{\labelenumi}{\text{\theenumi}}

\newcounter{aufgabe}
%\newenvironment{lsg}{\loesung}{}
\ifshow
  \newenvironment{lsg}{\loesung}{}
\else
  \excludecomment{lsg}
\fi

\newenvironment{inhalt}
  {\paragraph{Inhalt des Übungsblatts:}\itemize\let\origitem\item}
  {\enditemize\vspace{2em}}

\newcommand{\R}{\ensuremath\mathbb{R}}
\newcommand{\N}{\ensuremath\mathbb{N}}
\newcommand{\Z}{\ensuremath\mathbb{Z}}
\newcommand{\LM}{\ensuremath\mathbb{L}}
\newcommand{\intd}{\ensuremath\mathrm{d}}
\newcommand{\e}{\ensuremath\mathrm{e}}
\renewcommand{\d}{\,\mathrm{d}}
\newcommand{\stf}[1]{\ensuremath \left[ #1 \right]}

\newcommand{\cas}{\hfill (CAS)}
\newcommand{\seite}[1]{\textit{(S. #1)}}

\newcommand{\vektor}[1]{\ensuremath\begin{pmatrix} #1 \end{pmatrix}}


\everymath{\displaystyle}

%Malpunkte
\mathcode`\*="8000
{\catcode`\*\active\gdef*{\cdot}}

%SECTION
\titleformat{\section}
{\clearpage\setcounter{aufgabe}{0}\vspace{1em}\Large\raggedright\bfseries}
{}
{0pt}
{}

\titleformat{\subsection}[runin]
{\stepcounter{aufgabe}\vspace{1px}\normalfont\raggedright\bfseries}
{A\theaufgabe: }
{0pt}
{\ }

\titleformat{\subsubsection}[runin]
{\normalfont\raggedright\bfseries}
{Lösung \theaufgabe: }
{0pt}
{\ }


%FANCYHDR
\pagestyle{fancy}
\lhead{\small Simon König\\ Joshua Fabian}
\rhead{\small Mathecrashkurs 2018}
\cfoot{Seite \thepage\thinspace von\thinspace\pageref{LastPage}}
\lfoot{}
\renewcommand{\headrulewidth}{0.5pt}
\renewcommand{\footrulewidth}{0pt}

\title{Mathe-Crashkurs 2018 - Übungsblatt}
\date{\today}
\author{Simon König, Joshua Fabian}

\chead{\Large Übungsblatt 2}

\begin{document}
\begin{inhalt}
	\item Wachstum\seite{51}
	\item LGS-Rechnung \seite{57}, Vektorrechnung \seite{69}
	\item Geraden und Ebenen \seite{73}
\end{inhalt}




\aufgabe{Wachstum}\cas
\begin{enumerate}
	\item Bakterien vermehren sich durch Teilung, wobei sich eine Bakterienzelle durchschnittlich alle 10 Minuten teilt. Zum Zeitpunkt $t=0$ sei genau eine Bakterienzelle vorhanden.
	\begin{itemize}
		\item Wie viele Bakterien sind dann nach 1 Stunde, 2 Stunden bzw. 6 Stunden vorhanden?
		\item Finde eine Formel für die Anzahl $B(t)$ der Bakterien nach der Zeit $t$.
		\item Eine Bakterienzelle hat ein Volumen von ca. $2\times10^{-18}\mathrm{m}^3$ . Wie lange dauert es, bis die Bakterienkultur ein Volumen von $1\mathrm{m}^3$ bzw. $1\mathrm{km}^3$ einnimmt? Ist das Ergebnis plausibel?
	\end{itemize}
	\item	Angenommen, die Weltbevölkerung vermehrt sich nach der Formel $M(t)=M_0*\e^{\delta t}$.

	1960 gab es ca. 3 Milliarden Menschen, 1995 etwa 5,6 Milliarden.
	\begin{itemize}
		\item Bestimme die Konstante $\delta$.
		\item Wieviel Prozent beträgt das jährliche Wachstum der Weltbevölkerung?
		\item Wann wird die Erde 15 Mrd. Einwohner haben, wenn die Bevölkerung im selben Tempo weiterwächst?
	\end{itemize}
	\item Eine Tasse kochendheißer Kaffee (100$^\circ$C) kühlt bei Zimmertemperatur (20$^\circ$C) in 10 Minuten auf 30$^\circ$C ab.
	\begin{itemize}
		\item Geben Sie eine Funktionsgleichung für die Temperatur des Kaffees an.
		\item Frau M mischt den Kaffee mit der gleichen Menge Milch aus dem Kühlschrank (4$^\circ$C). Sie hat zwei Möglichkeiten: die Milch sofort dazugeben, danach 3 Minuten warten oder die Milch erst nach 3 Minuten dazugeben.\\
Welche Temperatur hat der Milchkaffee in beiden Fällen?
\textit{(Hinweis: Die Temperatur der Mischung ist der Mittelwert der einzelnen Temperaturen: $T=\frac{(T_1+T_2)}{2}$.)}
	\end{itemize}
\end{enumerate}
\begin{lsg}{}
	\begin{enumerate}
		\item
		\begin{itemize}
			\item Exponentielles Wachstum, allgemeine Form: $B(t)=c*a^x=c*\e^{\ln(a)x}$%
			\begin{align*}
				c&=1, \text{ da $B(0)=1$}\\
				\intertext{Es soll gelten:}
				B(10)&=2=a^{10}\\
				a&=\sqrt[10]{2}\\
				a&\approx 1,072
				\intertext{Bakterien nach einer, zwei und sechs Studen:}
				B(60)&=1,072^{60}\approx 64\\
				B(120)&=1,072^{120}\approx 4201\\
				B(360)&=1,072^{360}\approx 74\times 10^{9}
			\end{align*}
			\item $B(t)=1*1,072^t=\e^{\ln(1,072)t}$
			\item Es soll gelten:
			\begin{align*}
				2\times10^{-18}\mathrm{m}^3 * B(n) &= 1\mathrm{m}^3 & 2\times10^{-18}\mathrm{m}^3 * B(n) &= 1000000000\mathrm{m}^3\\
				\rightsquigarrow t&\approx 586,16 \hat=9,77\mathrm h & \rightsquigarrow t&\approx 884,23\hat=14,74\mathrm h
			\end{align*}
		\end{itemize}
		\item
		\begin{itemize}
			\item \begin{align*}
				M(t)&=M_0*\e^{\delta t}\\
				\intertext{$t=0$: 1960}
				M(0)&=M_0=3\times10^9\\
				\intertext{$t=1995-1960=35$}
				M(35)&=3\times10^{9}*\e^{\delta 35}=5,6\times 10^9\\
				\delta&=0,017833
			\end{align*}
			\item Jährliches Wachstum: $\e^\delta$
			\begin{align*}
				\e^\delta = \e^{0,017833}=1,01799\ \hat=\ 101,8\%
			\end{align*}
			\item \begin{align*}
				M(t)&=3\times10^{9}*\e^{0,017833*t}=15\times 10^9\\
				\rightsquigarrow t&= 90\ \hat=\ 1960+90=2050
			\end{align*}
		\end{itemize}
		\item\begin{itemize}
			\item Beschränktes Wachstum, allgemein: $T(t)=S-(S-T(0))*\e^{-k*t}$
			\begin{align*}
				&S=20, S-T(0)=20-100=-80\\
				\Rightarrow T(t)&=20+80*\e^{-k*t}\\
				\intertext{Berechnung von $k$ durch einsetzen der bekannten Werte:}
				T(10)&=30=20+80*\e^{-k*10} \quad| \text{ Auflösen nach k}\\
				\rightsquigarrow k&=\frac{3*\ln(2)}{10}\\
				\rightsquigarrow T(t)&=20+80*\e^{-\frac{3\ln(2)}{10}*t}
			\end{align*}
			\item Erst abkühlen, dann Milch dazu: $33,44^\circ\mathrm{C}$
			Variante 2: $37,14^\circ\mathrm{C}$
		\end{itemize}
	\end{enumerate}
\end{lsg}




\aufgabe{Geraden und Ebenen aufstellen: }

	\begin{enumerate}
		\item Gib eine Geradengleichung so an, dass sie durch die gegebenen Punkte läuft:
		\begin{itemize}
			\item $P(1|3|5)$, $Q(2|2|2)$
			\item $R(-2|2|0)$, $S(1|3|4)$
		\end{itemize}
		\item Berechne eine Ebenengleichung, die senkrecht zur Gerade $g$ liegt und die den Punkt $P(5|5|5)$ enthält.
		\begin{equation*}
			g:\vec x=\vektor{1\\3\\5}+t*\vektor{1\\-1\\-3}
		\end{equation*}
		\item Gegeben sind die Gerade $h$ und der Punkt $P(2|2|2)$, gib eine Ebenengleichung an, die sowohl die Gerade als auch den Punkt enthält.
		\begin{equation*}
			h:\vec x=\vektor{2\\5\\1}+t*\vektor{6\\2\\1}
		\end{equation*}
		\item Gib einen Vektor an, der senkrecht zu den beiden angegebenen Vektoren ist. Berechne außerdem den Flächeninhalt des durch die gegebenen Vektoren aufgespannten Parallelogramms.
		\begin{itemize}
			\item$\vec u=\vektor{1\\5\\3},\vec v=\vektor{2\\1\\3}$
			\item$\vec r=\vektor{6\\3\\7},\vec s=\vektor{1\\2\\4}$
		\end{itemize}
	\end{enumerate}

\begin{lsg}{}
	\begin{enumerate}
		\item
		\begin{itemize}
			\item $g:\vec x=\vektor{1\\3\\5}+r*\vektor{1\\-1\\-3}$
			\item $g:\vec x=\vektor{-2\\2\\0}+r*\vektor{3\\1\\-4}$
		\end{itemize}
		\item $E: \left[\vec x-\vektor{5\\5\\5}\right]*\vektor{1\\-1\\-3}=0$
		\item Zwei Punkte $H_t$ berechnen, die auf der Geraden liegen: $H_0(2|5|1)$, $H_1(8|7|2)$

		Berechnen der Richtungsvektoren von $P$ zu den zwei Punkten:
		\begin{align*}
			\overrightarrow{PH_0}&=\vektor{0\\3\\-1}\\
			\overrightarrow{PH_1}&=\vektor{6\\5\\0}
		\end{align*}
		Damit ist die Ebenengleichung mit $P$ als Stützpunkt:
		\begin{equation*}
			E:\vec x=\vektor{2\\2\\2}+r*\vektor{0\\3\\-1}+s*\vektor{6\\5\\0}
		\end{equation*}
		\item
		\begin{enumerate}
			\item Kreuzprodukt: $\vec u\times \vec v=\vektor{5*3-3\\3*2-3\\1-5*2}=\vektor{12\\3\\-9}$
			(Alternativ auch durch LGS berechenbar, zwei Gleichungen mit dem Skalarprodukt aufstellen.)

			Nachweis durch Skalarprodukt mit beiden Vektoren, jedes mal muss Null rauskommen.

			$12+3*5-9*3=0$, $12*2+3-9*3=0$

			Flächeninhalt des aufgespannten Parallelogramms ist gleich dem Betrag des Vektors, der durch das Kreuzprodukt entsteht:

			$A=\left|\vektor{12\\3\\-9}\right|=\sqrt{144+9+81}=\sqrt{234}\approx 15,30$
			\item Kreuzprodukt: $\vec r\times \vec s=\vektor{3*4-7*2\\7-6*4\\6*2-3}=\vektor{-2\\-17\\9}$

			Nachweis durch Skalarprodukt mit beiden Vektoren, jedes mal muss Null rauskommen.

			$-2*6-17*3+9*7=0$, $-2-17*2+9*4=0$

			Flächeninhalt des aufgespannten Parallelogramms ist gleich dem Betrag des Vektors, der durch das Kreuzprodukt entsteht:

			$A=\left|\vektor{-2\\-17\\9}\right|=\sqrt{4+289+81}=\sqrt{374}\approx 19,34$
		\end{enumerate}
	\end{enumerate}
\end{lsg}





\aufgabe{\color{red}Ebenengleichungen: }
Bestimme die Ebene in der angegebenen Darstellungsform:
\begin{enumerate}
	\item $E$ enthält die Punkte $A(2|2|2), B(4|1|3)$ und $C(8|4|5)$. Gib $E$ in Normalenform an. %Parameterform -> Normalenform,
	\item Die gesuchte Ebene $F$ ist die Spiegelebene zwischen $A(1|4|7)$ und $A'(3|2|3)$. Gib $F$ in Parameterform an. %Normalenform -> Parameterform
	\item Die Ebene $G$ enthält die Gerade $\vec x = \vektor{3\\1\\2}+s*\vektor{2\\0\\-1}$ und ist orthogonal zur Ebene $H:-x_1+x_2+2x_3+2=0$. Gib die Ebene $G$ in Koordinatenform an. %Parameterform -> Koordinatenform
\end{enumerate}
\begin{lsg}{}
	\begin{enumerate}
		\item Parameterform von $E$: $\vec{x}=\overrightarrow{OA}+s*\overrightarrow{AB}+t*\overrightarrow{AC}\rightsquigarrow \vec{x}=\vektor{2\\2\\2}+s*\vektor{2\\-1\\1}+t*\vektor{6\\2\\3}$\\
		Umformen nach Normalenform: $\vektor{2\\-1\\1}\times \vektor{6\\2\\3}=\vektor{-5\\0\\10}$\\
		$E:\left[\vec{x}-\vektor{2\\2\\2}\right]*\vektor{-5\\0\\10}=0$
		\item In Normalenform aufstellen:
		\begin{equation*}
			F: \left[\vec x - \frac{1}{2}*\overrightarrow{AA'}+\overrightarrow{OA}\right]*\overrightarrow{AA'}=0\rightsquigarrow F: \left[\vec x - \vektor{2\\3\\5}\right]*\vektor{2\\-2\\-4}=0
		\end{equation*}
		In Parameterform umwandeln: 
		\item 
	\end{enumerate}
\end{lsg}




\aufgabe{Lineare Gleichungssysteme: }
Gib die Lösungsmenge der Gleichungssysteme an:
\begin{multicols}{2}
	\begin{enumerate}
		\item\begin{alignat*}{4}%
			-2x_1& -x_2& - x_3& = -4\\
			2x_1& +2x_2 & + x_3& = -8\\
			-x_1& -3x_2 & + x_3& =-15
		\end{alignat*}
		\item\begin{alignat*}{4}%
			2x&-3y&=19\\
			4x&-8z&=20\\
			5y&-4z&=-7
		\end{alignat*}
	\end{enumerate}
\end{multicols}
\begin{lsg}{}
\begin{enumerate}
	\item Zeile 1 auf Zeilen 2 und 3 addieren, dann ergibt sich sofort $x_2=4$. $x_2$ in die dritte Zeile einsetzen $\rightarrow x_1=1$. Beide Werte in die erste Gleichung einsetzen, $x_3=-2$.

	$\mathbb{L}=\simpleset{(1,4,-2)}$
	\item Aufstellen einer geeigneten Matrix
	\begin{align*}
		&\ematrix{rrr|r}{2&-3&0&19\\4&0&-8&20\\0&5&-4&-7}
		\rightsquigarrow\ematrix{rrr|r}{2&-3&0&19\\0&-6&8&18\\0&5&-4&-7}\rightsquigarrow\ematrix{rrr|r}{2&-3&0&19\\0&-6&8&18\\0&0&16&48}\\
		\rightsquigarrow&\mathbb{L}=\simpleset{(11,1,3)}
	\end{align*}
\end{enumerate}
\end{lsg}

\end{document}
