\input{master/layout}
\chead{\Large Simon Aufgabensammlung}


\begin{document}



\aufgabe{Ebenengleichungen: }
Bestimme die Ebene in der angegebenen Darstellungsform:
\begin{enumerate}
	\item $E$ enthält die Punkte $A(2|2|2), B(4|1|3)$ und $C(8|4|5)$. Gib $E$ in Normalenform an. %Parameterform -> Normalenform,
	\item Die gesuchte Ebene $F$ ist die Spiegelebene zwischen $A(1|4|7)$ und $A'(3|2|3)$. Gib die $F$ in Parameterform an. %Normalenform -> Parameterform
	\item Die Ebene $G$ enthält die Gerade $\vec x = \vektor{3\\1\\2}+s*\vektor{2\\0\\-1}$ und ist orthogonal zur Ebene $H:-x_1+x_2+2x_3+2=0$. Gib die Ebene $G$ in Koordinatenform an. %Parameterform -> Koordinatenform
\end{enumerate}

\aufgabe{Schnittgerade: }
Gib eine Gleichung der Schnittgeraden der Ebenen $E: x_1-x_2+2x_3=7$ und $F:6x_1+x_2-x_3+7=0$ an.

\aufgabe{Abstandsberechnungen Teil 1: }
Berechne den Abstand des Punktes $R(6|9|4)$ von der Ebene $E: \left[ \vec x -  \vektor{7 \\ 5\\ 2}\right]*\vektor{2\\ 2 \\1}=0$



\aufgabe{Abstands- und Lageberechnungen: }
Gegeben sind die Ebene \\$E: \left[ \vec x -  \vektor{-1 \\ 4\\ -3}\right]*\vektor{8\\ 1 \\-4}=0$ und die Gerade $g: \vec x=\vektor{7\\5\\-7}+t*\vektor{1\\-4\\1}$.
\begin{enumerate}
	\item Zeigen Sie, dass $E$ und $g$ parallel zueinander sind.
	\item Bestimmen Sie den Abstand von $E$ und $g$.
\end{enumerate}

\begin{lsg}{}
	\begin{enumerate}
		\item Dafür müssen der Normalenvektor von $E$ und der Richtungsvektor von $g$ orthogonal sein.
		\begin{equation*}
			\vektor{8\\ 1 \\-4}*\vektor{1\\-4\\1}=8-4-4=0
		\end{equation*}
		\item Aufstellen einer Hilfsgeraden:
		\begin{align*}
			h: \vec x=\vektor{7\\5\\-7}+r*\vektor{8\\ 1 \\-4}
		\end{align*}
	\end{enumerate}
\end{lsg}





\aufgabe{Anwedungsaufgaben: }
  \begin{enumerate}
    \item Gegeben sei die Funktion $f(x) = \frac{1-4x^2}{x^2}$. Ihr Schaubild ist $K$, wo schneidet die Tangente an $K$ in $P(1|f(1))$ die $x$-Achse?
		\item
  \end{enumerate}
	\begin{lsg}{}
		\begin{enumerate}
			\item Aufstellen der allgemeinen Tangentengleichung für $f$:
			\begin{align*}
				t_f(x)&=f'(x_0)*(x-x_0)+f(x_0)\\
							&=\left(\frac{-2}{x_0^3}\right)*(x-x_0)+\frac{1}{x_0^2}-4\\
				\intertext{Für $x_0=1$ folgt:}
				t_{f,1}&=-2(x-1)-3\\
				\intertext{Finden der Nullstelle von $t_{f,1}$:}
				t_{f,1}&\overset!=0=-2(x-1)-3\\
			\end{align*}
		\end{enumerate}
	\end{lsg}



	\aufgabe{Uneigentliches Flächenintegral}
	\begin{enumerate}
		\item Berechnen Sie $\int\limits^\infty_0 e^{-x}\d x$
		\item	Bestimmen Sie $\int\limits^\infty_0 x^a\d x$ in Abhängigkeit von $a$ mit $a < -1$.
		\item Berechnen Sie $\lim\limits_{x\rightarrow -\infty} x^5\e^x$
	\end{enumerate}
	\begin{lsg}{}
		\begin{enumerate}
			\item \begin{align*}
				&\int\limits^k_0 e^{-x}\d x\\
				&=\left[-\e^{-x}\right]_0^k\\
				&=(-\e^{-k})-(-1)\\
				&=-\e^{-k}+1\\
				&\lim\limits_{k\rightarrow \infty}-\e^{-k}+1\\
				&=0+1=1
			\end{align*}
			\item \begin{align*}
				&\int\limits^\infty_0 x^a\d x\\
				&=\left[\frac1 a x^{a+1}\right]_0^k\\
				&=\left(\frac1 k k^{a+1}\right)-\left(\frac1 a 0^{a+1}\right)\\
				&=\left(\frac1 k k^{a+1}\right)\\
				&\lim\limits_{k\rightarrow \infty}\frac1 k k^{a+1}\\
				&=0\quad \text{da $k^{a+1}<1$, denn $a<-1$}
			\end{align*}
		\end{enumerate}
	\end{lsg}



	\aufgabe{Gebrochenrationale Funktionen}
	\begin{enumerate}
		\item Gegeben sei die Funktion $f(x)=\frac{x}{x^2-9}$. Skizzieren Sie den Graphen und geben Sie den Definitionsbereich $D_f$ an.
		\item	Geben Sie die Funktionsgleichungen aller waagerechter oder senkrechter Asymptoten von $f(x)=\frac{4x^2+3}{x^2+5x}$
	\end{enumerate}



	\aufgabe{LGS}
	%Nutzen in {alignat}{5}:{NR}{K1}{K2}{K3}{Ergebnis}
	\newcommand{\lgslinethree}[4]{#1 x\quad &+&  #2 y\quad &+&  #3 z\quad & = #4 \\}
	\newcommand{\lgslinetwo}[3]{#1 x\quad &+&  #2 y\quad & = #3 \\}
	Berechnen Sie die Lösungsmenge $\LM\subseteq\R$ der linearen Gleichungssysteme:
	\begin{multicols}{2}
		\begin{enumerate}
			\item
			\begin{alignat*}{5}
				\lgslinethree{}{3}{}{1}
				\lgslinethree{3}{9}{4}{5}
				\lgslinethree{}{3}{2}{3}
			\end{alignat*}
			\item
			\begin{alignat*}{4}
				\lgslinethree{6}{}{}{31}
				\lgslinethree{3}{1}{4\frac 1 2}{19}
				\lgslinethree{-2}{-1}{0}{-15}
			\end{alignat*}
			\item
			\begin{alignat*}{4}
				\lgslinetwo{3}{2}{1}
				\lgslinetwo{1}{-2}{11}
				\lgslinetwo{-2}{}{5}
			\end{alignat*}
		\end{enumerate}
	\end{multicols}



\aufgabe{Funktionsterm aufstellen: }
\begin{lsg}{}

\end{lsg}


\aufgabe{Vektorrechnung}
\begin{enumerate}
	\item Berechne das Skalarprodukt und das Kreuzprodukt der beiden Vektoren $\vektor{1\\4\\-2}$ und $\vektor{-4\\5\\2}$.
	\item Berechne $\alpha$ so, dass die beiden Vektoren $\vektor{\\\\}$ und $\vektor{\\\\}$ orthogonal zueinander sind.
	\item
\end{enumerate}
\begin{lsg}{}
	\begin{enumerate}
		\item \begin{align*}
			\vektor{1\\4\\-2}*\vektor{-4\\5\\2}&=-4+20-4=12 \\
			\vektor{1\\4\\-2}\times\vektor{-4\\5\\2}&=\vektor{8+10\\8-2\\5+16}=\vektor{18\\6\\21}
		\end{align*}
		\item Berechne $\alpha$ so, dass die beiden Vektoren $\vektor{\\\\}$ und $\vektor{\\\\}$ orthogonal zueinander sind.
		\item
	\end{enumerate}
\end{lsg}


\end{document}
