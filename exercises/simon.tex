\documentclass[a4paper, oneside]{article}
\usepackage[utf8]{inputenc}
\usepackage[ngerman]{babel}
\usepackage[top=2.5cm, bottom=3cm, outer=2.5cm, inner=2.5cm, heightrounded]{geometry}
\usepackage{graphicx}
\usepackage{morefloats}
\usepackage{wrapfig}
\usepackage{hyperref}
\usepackage{cite}
\usepackage{siunitx}
\usepackage[default]{sourcesanspro}
\usepackage[T1]{fontenc}
\usepackage{url}
\usepackage{marginnote}
\usepackage[font=footnotesize]{caption}
\usepackage{color}
\usepackage{xcolor}
\usepackage{multicol}
\usepackage[fleqn]{mathtools}
\usepackage{amssymb}
\usepackage{wrapfig}
\usepackage[noindentafter]{titlesec}
\usepackage{fancyhdr}
\usepackage{lastpage}
\usepackage{comment}

%% LÖSUNGEN ANZEIGEN
\newif\ifshow
%\showtrue
\showfalse

%%%SECTIONING
\renewcommand*{\marginfont}{\noindent\rule{0pt}{0.7\baselineskip}\footnotesize}

\newcommand{\aufgabe}[1]{\subsection{#1}}
\newcommand{\loesung}[1]{\subsubsection{#1}}

\newcommand{\simpleset}[1]{\ensuremath \left\{ #1 \right\}}
\newcommand{\ematrix}[2]{\renewcommand{\arraystretch}{1}\ensuremath\left(\begin{array}{@{}#1@{}}#2\end{array}\right)}

\renewcommand{\theenumi}{\alph{enumi})}
\renewcommand{\labelenumi}{\text{\theenumi}}

\newcounter{aufgabe}
%\newenvironment{lsg}{\loesung}{}
\ifshow
  \newenvironment{lsg}{\loesung}{}
\else
  \excludecomment{lsg}
\fi

\newenvironment{inhalt}
  {\paragraph{Inhalt des Übungsblatts:}\itemize\let\origitem\item}
  {\enditemize\vspace{2em}}

\newcommand{\R}{\ensuremath\mathbb{R}}
\newcommand{\N}{\ensuremath\mathbb{N}}
\newcommand{\Z}{\ensuremath\mathbb{Z}}
\newcommand{\LM}{\ensuremath\mathbb{L}}
\newcommand{\intd}{\ensuremath\mathrm{d}}
\newcommand{\e}{\ensuremath\mathrm{e}}
\renewcommand{\d}{\,\mathrm{d}}
\newcommand{\stf}[1]{\ensuremath \left[ #1 \right]}

\newcommand{\cas}{\hfill (CAS)}
\newcommand{\seite}[1]{\textit{(S. #1)}}

\newcommand{\vektor}[1]{\ensuremath\begin{pmatrix} #1 \end{pmatrix}}


\everymath{\displaystyle}

%Malpunkte
\mathcode`\*="8000
{\catcode`\*\active\gdef*{\cdot}}

%SECTION
\titleformat{\section}
{\clearpage\setcounter{aufgabe}{0}\vspace{1em}\Large\raggedright\bfseries}
{}
{0pt}
{}

\titleformat{\subsection}[runin]
{\stepcounter{aufgabe}\vspace{1px}\normalfont\raggedright\bfseries}
{A\theaufgabe: }
{0pt}
{\ }

\titleformat{\subsubsection}[runin]
{\normalfont\raggedright\bfseries}
{Lösung \theaufgabe: }
{0pt}
{\ }


%FANCYHDR
\pagestyle{fancy}
\lhead{\small Simon König\\ Joshua Fabian}
\rhead{\small Mathecrashkurs 2018}
\cfoot{Seite \thepage\thinspace von\thinspace\pageref{LastPage}}
\lfoot{}
\renewcommand{\headrulewidth}{0.5pt}
\renewcommand{\footrulewidth}{0pt}

\title{Mathe-Crashkurs 2018 - Übungsblatt}
\date{\today}
\author{Simon König, Joshua Fabian}

\chead{\Large Simon Aufgabensammlung}


\begin{document}



\aufgabe{Ebenengleichungen: }
Bestimme die Ebene in der angegebenen Darstellungsform:
\begin{enumerate}
	\item $E$ enthält die Punkte $A(2|2|2), B(4|1|3)$ und $C(8|4|5)$. Gib $E$ in Normalenform an. %Parameterform -> Normalenform,
	\item Die gesuchte Ebene $F$ ist die Spiegelebene zwischen $A(1|4|7)$ und $A'(3|2|3)$. Gib $F$ in Parameterform an. %Normalenform -> Parameterform
	\item Die Ebene $G$ enthält die Gerade $\vec x = \vektor{3\\1\\2}+s*\vektor{2\\0\\-1}$ und ist orthogonal zur Ebene $H:-x_1+x_2+2x_3+2=0$. Gib die Ebene $G$ in Koordinatenform an. %Parameterform -> Koordinatenform
\end{enumerate}
\begin{lsg}{}
	\begin{enumerate}
		\item
		\item In Normalenform aufstellen:
		\begin{equation*}
			F: \left[\vec x - \frac 12 *\vektor{2\\-2\\-4}\right]*\vektor{2\\-2\\-4}=0
		\end{equation*}
	\end{enumerate}
\end{lsg}




%FERTIG
\aufgabe{Differentialrechnung: }
\begin{enumerate}
	\item Gegeben sei folgende Funktion: $F(x,y)=2x^3-5y+x^2+10x-10$. Bestimmen Sie die Ableitung der durch $F(x,y)=0$ implizit gegebenen Funktion $y=h(x)$.
	\item Gegeben sind die Funktionen:
	$f(x) = (u \circ v)(x)$ und $g(x) = (u* v)(x)$
	Bestimme die Ableitungen von $f$ und $g$ für $u(x)=x^2$ und $v(x)=\sin(2x)$
	\item Bestimme jeweils $f_i'(3)$
	\begin{itemize}
		\item $f_1(x) = (x+5)^2$
		\item $f_2(x) = \frac{1}{(x-5)^2}$
	\end{itemize}
\end{enumerate}
\begin{lsg}{}
	\begin{enumerate}
		\item Aus $F(x,y)=0$ und $y=h(x)$ folgt:\begin{alignat*}{2}
		&2x^3-5y+10x-10=0\quad&&|\ \text{Nach $y=$ umformen}\\
		&y=\frac{2}{5} x^3+\frac{1}{5} x^2+2x-2 &&\Rightarrow h(x)\\
		&h'(x)=\frac{6}{5} x^2+\frac{2}{5}x+2
		\end{alignat*}
		\item \begin{align*}
		f'(x)&=\left(u(v(x))\right)'=\left({\sin(2x)}^2\right)'=2\sin(2x)\cdot \cos(2x) \cdot 2=4\sin(2x)\cdot \cos(2x)\\
		g'(x)&=(u(x)\cdot v(x))'=\left(x^2\cdot \sin(2x)\right)'=2x\cdot \sin(2x)+x^2\cdot 2\cos(2x)
		\end{align*}
		\item \begin{align*}
		f_1'(x)&=2(x+5)&\Rightarrow  &f_1'(3)=16\\
		f_2'(x)&=\frac{-2}{(x-5)^3}&\Rightarrow &f_2'(3)=\frac{1}{4}
		\end{align*}
	\end{enumerate}
\end{lsg}


\aufgabe{Abstandsberechnungen Teil 1: }
Berechne den Abstand des Punktes $R(6|9|4)$ von der Ebene $E: \left[ \vec x -  \vektor{7 \\ 5\\ 2}\right]*\vektor{2\\ 2 \\1}=0$









\aufgabe{Anwedungsaufgaben: }
  \begin{enumerate}
    \item Gegeben sei die Funktion $f(x) = \frac{1-4x^2}{x^2}$. Ihr Schaubild ist $K$, wo schneidet die Tangente an $K$ in $P(1|f(1))$ die $x$-Achse?
		\item
  \end{enumerate}
	\begin{lsg}{}
		\begin{enumerate}
			\item Aufstellen der allgemeinen Tangentengleichung für $f$:
			\begin{align*}
				t_f(x)&=f'(x_0)*(x-x_0)+f(x_0)\\
							&=\left(\frac{-2}{x_0^3}\right)*(x-x_0)+\frac{1}{x_0^2}-4\\
				\intertext{Für $x_0=1$ folgt:}
				t_{f,1}&=-2(x-1)-3\\
				\intertext{Finden der Nullstelle von $t_{f,1}$:}
				t_{f,1}&\overset!=0=-2(x-1)-3\\
			\end{align*}
		\end{enumerate}
	\end{lsg}



	\aufgabe{Uneigentliches Flächenintegral}
	\begin{enumerate}
		\item Berechnen Sie $\int\limits^\infty_0 e^{-x}\d x$
		\item	Bestimmen Sie $\int\limits^\infty_0 x^a\d x$ in Abhängigkeit von $a$ mit $a < -1$.
	\end{enumerate}
	\begin{lsg}{}
		\begin{enumerate}
			\item \begin{align*}
				&\int\limits^k_0 e^{-x}\d x\\
				&=\left[-\e^{-x}\right]_0^k\\
				&=(-\e^{-k})-(-1)\\
				&=-\e^{-k}+1\\
				&\lim\limits_{k\rightarrow \infty}-\e^{-k}+1\\
				&=0+1=1
			\end{align*}
			\item \begin{align*}
				&\int\limits^\infty_0 x^a\d x\\
				&=\left[\frac1 a x^{a+1}\right]_0^k\\
				&=\left(\frac1 k k^{a+1}\right)-\left(\frac1 a 0^{a+1}\right)\\
				&=\left(\frac1 k k^{a+1}\right)\\
				&\lim\limits_{k\rightarrow \infty}\frac1 k k^{a+1}\\
				&=0\quad \text{da $k^{a+1}<1$, denn $a<-1$}
			\end{align*}
		\end{enumerate}
	\end{lsg}



	\aufgabe{Gebrochenrationale Funktionen}
	\begin{enumerate}
		\item Gegeben sei die Funktion $f(x)=\frac{x}{x^2-9}$. Skizzieren Sie den Graphen und geben Sie den Definitionsbereich $D_f$ an.
		\item	Geben Sie die Funktionsgleichungen aller waagerechter oder senkrechter Asymptoten von $f(x)=\frac{4x^2+3}{x^2+5x}$
	\end{enumerate}



	\aufgabe{LGS}
	%Nutzen in {alignat}{5}:{NR}{K1}{K2}{K3}{Ergebnis}
	\newcommand{\lgslinethree}[4]{#1 x\quad &+&  #2 y\quad &+&  #3 z\quad & = #4 \\}
	\newcommand{\lgslinetwo}[3]{#1 x\quad &+&  #2 y\quad & = #3 \\}
	Berechnen Sie die Lösungsmenge $\LM\subseteq\R$ der linearen Gleichungssysteme:
	\begin{multicols}{2}
		\begin{enumerate}
			\item
			\begin{alignat*}{5}
				\lgslinethree{}{3}{}{1}
				\lgslinethree{3}{9}{4}{5}
				\lgslinethree{}{3}{2}{3}
			\end{alignat*}
			\item
			\begin{alignat*}{4}
				\lgslinethree{6}{}{}{31}
				\lgslinethree{3}{1}{4\frac 1 2}{19}
				\lgslinethree{-2}{-1}{0}{-15}
			\end{alignat*}
			\item
			\begin{alignat*}{4}
				\lgslinetwo{3}{2}{1}
				\lgslinetwo{1}{-2}{11}
				\lgslinetwo{-2}{}{5}
			\end{alignat*}
		\end{enumerate}
	\end{multicols}



\aufgabe{Vektorrechnung}
\begin{enumerate}
	\item Berechne das Skalarprodukt und das Kreuzprodukt der beiden Vektoren $\vektor{1\\4\\-2}$ und $\vektor{-4\\5\\2}$.
	\item Berechne $\alpha$ so, dass die beiden Vektoren $\vektor{\\\\}$ und $\vektor{\\\\}$ orthogonal zueinander sind.
	\item
\end{enumerate}
\begin{lsg}{}
	\begin{enumerate}
		\item \begin{align*}
			\vektor{1\\4\\-2}*\vektor{-4\\5\\2}&=-4+20-4=12 \\
			\vektor{1\\4\\-2}\times\vektor{-4\\5\\2}&=\vektor{8+10\\8-2\\5+16}=\vektor{18\\6\\21}
		\end{align*}
		\item Berechne $\alpha$ so, dass die beiden Vektoren $\vektor{\\\\}$ und $\vektor{\\\\}$ orthogonal zueinander sind.
		\item
	\end{enumerate}
\end{lsg}


\end{document}
