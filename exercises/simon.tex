\input{master/layout}
\chead{\Large Simon Aufgabensammlung}


\begin{document}






\aufgabe{Wurzeln: } Vereinfache so weit möglich
\begin{multicols}{3}
  \begin{enumerate}
    \item $4\sqrt{7}-6\sqrt{7} + 3\sqrt{7}$
    \item $\sqrt{180}$
    \item $\sqrt{192x^5}$
    \item $\sqrt{\frac{80x^2}{147}}$
    \item $\sqrt{3}\cdot\sqrt{300}$
    \item $\sqrt{27}-\sqrt{25}-\sqrt{3}$
  \end{enumerate}
\end{multicols}

\begin{lsg}{}
  \begin{multicols}{3}
    \begin{enumerate}
      \item $\sqrt{7}$
      \item $\sqrt{4*45}=2*\sqrt{5*9}=6*\sqrt 5$
      \item $\sqrt{192*x^5}=\sqrt{192}*\sqrt{x^5}=\sqrt{2*96}*\sqrt{x^2*x^2*x}=4*x^2*\sqrt{12x}$
      \item $\frac{\sqrt{80x^2}}{\sqrt{147}} = \frac{4x*\sqrt{5}}{7*\sqrt{3}}$
      \item $\sqrt{3}*\sqrt{300}=3*\sqrt{100}=30$
      \item $\sqrt{3*9}-5-\sqrt{3}=3*\sqrt{3}-5-\sqrt{3}=2*\sqrt{3}-5$
    \end{enumerate}
  \end{multicols}
\end{lsg}



\aufgabe{Monotonieuntersuchung: }
\begin{enumerate}
	\item Definiere, wann eine Funktion in einem Intervall als streng monoton steigend bezeichnet wird.
	\item Untersuche die gegebenen Funktionen auf Monotonie und gib Art und Lage der Extrempunkte an:
	\begin{enumerate}
		\item $f(x)=x^2-3x+5$
		\item
	\end{enumerate}
\end{enumerate}
\begin{lsg}{}
	\begin{enumerate}
		\item Eine Funktion ist über dem Intervall $[x_0;x_1]$ streng monoton steigend, wenn gilt: $f'(x)>0$ für alle $x_0\leq x \leq x_1$
		\item
		\begin{enumerate}
			\item
		\end{enumerate}
	\end{enumerate}
\end{lsg}










\aufgabe{Nullstellen von Funktionen: } Bestimme die Nullstellen der Funktionen
\begin{multicols}{3}
  \begin{enumerate}
    \item $f(x) = \frac 1 2 (x-2)^2 -4$
    \item $f(x) = 3x^2-2x+2$
    \item $f(x) = x^4-1$
  \end{enumerate}
\end{multicols}

\begin{lsg}{}
  \begin{multicols}{3}
    \begin{enumerate}
      \item \begin{align*}
        \frac 1 2 (x-2)^2 &= 4\\
        (x-2)^2 &= 8\\
        \pm(x-2)&=\sqrt 8\\
        \rightarrow x_1&=2+\sqrt 8\\
        \rightarrow x_2&=2-\sqrt 8\\
      \end{align*}
      \columnbreak
      \item Mit der Mitternachtsformel: Diskriminante ist negativ, d.h. keine Lösungen
      \columnbreak
      \item Mit Substitution: $u\coloneqq x^2$, $u_1=1, u_2=-1$

			Resubstitution: $u=x^2 \rightsquigarrow x_1=1$
    \end{enumerate}
  \end{multicols}
\end{lsg}






\aufgabe{Anwedungsaufgaben: }
  \begin{enumerate}
    \item Gegeben sei die Funktion $f(x) = \frac{1-4x^2}{x^2}$. Ihr Schaubild ist $K$, wo schneidet die Tangente an $K$ in $P(1|f(1))$ die $x$-Achse?
		\item
  \end{enumerate}
	\begin{lsg}{}
		\begin{enumerate}
			\item Aufstellen der allgemeinen Tangentengleichung für $f$:
			\begin{align*}
				t_f(x)&=f'(x_0)*(x-x_0)+f(x_0)\\
							&=\left(\frac{-2}{x_0^3}\right)*(x-x_0)+\frac{1}{x_0^2}-4\\
				\intertext{Für $x_0=1$ folgt:}
				t_{f,1}&=-2(x-1)-3\\
				\intertext{Finden der Nullstelle von $t_{f,1}$:}
				t_{f,1}&\overset!=0=-2(x-1)-3\\
			\end{align*}
		\end{enumerate}
	\end{lsg}



	\aufgabe{Uneigentliches Flächenintegral}
	\begin{enumerate}
		\item Berechnen Sie $\int\limits^\infty_0 e^{-x}\d x$
		\item	Bestimmen Sie $\int\limits^\infty_0 x^a\d x$ in Abhängigkeit von $a$ mit $a < -1$.
		\item Berechnen Sie $\lim\limits_{x\rightarrow -\infty} x^5\e^x$
	\end{enumerate}

	\aufgabe{Gebrochenrationale Funktionen}
	\begin{enumerate}
		\item Gegeben sei die Funktion $f(x)=\frac{x}{x^2-9}$. Skizzieren Sie den Graphen und geben Sie den Definitionsbereich $D_f$ an.
		\item	Geben Sie die Funktionsgleichungen aller waagerechter oder senkrechter Asymptoten von $f(x)=\frac{4x^2+3}{x^2+5x}$
	\end{enumerate}

	\aufgabe{Wachstum}
	\begin{enumerate}
		\item Bakterien vermehren sich durch Teilung, wobei sich eine Bakterienzelle durchschnittlich alle 10 Minuten teilt. Zum Zeitpunkt $t=0$ sei genau eine Bakterienzelle vorhanden.
		\begin{itemize}
			\item Wie viele Bakterien sind dann nach 1 Stunde, 2 Stunden, 6 Stunden, 12 Stunde bzw. 24 Stunden vorhanden?
			\item Finde eine Formel für die Anzahl $B(t)$ der Bakterien nach der Zeit $t$.
			\item Eine Bakterienzelle hat ein Volumen von ca. $2\times10^{-18}\mathrm{m}^3$ . Wie lange dauert es, bis die Bakterienkultur ein Volumen von $1\mathrm{m}^3$ bzw. $1\mathrm{km}^3$ einnimmt? Ist das Ergebnis plausibel?
		\end{itemize}
		\item	Angenommen, die Weltbevölkerung vermehrt sich nach der Formel $M(t)=M_0*\e^{\delta t}$. 1960 gab es ca. 3 Milliarden Menschen, 1995 etwa 5,6 Milliarden.
		\begin{itemize}
			\item Bestimme die Konstante $\delta$.
			\item Wieviel Prozent beträgt das jährliche Wachstum der Weltbevölkerung?
			\item Wann wird die Erde 15 Mrd. Einwohner haben, wenn die Bevölkerung im selben Tempo weiterwächst?
		\end{itemize}
		\item Eine Tasse kochendheißer Kaffee (100$^\circ$C) kühlt bei Zimmertemperatur (20$^\circ$C) in 10 Minuten auf 30$^\circ$C ab.
		\begin{itemize}
			\item Geben Sie eine Funktionsgleichung für die Temperatur des Kafffees an.
			\item Frau M mischt den Kaffee mit der gleichen Menge Milch aus dem Kühlschrank (4$^\circ$C). Sie hat zwei Möglichkeiten: die Milch sofort dazugeben, danach 3 Minuten warten oder die Milch erst nach 3 Minuten dazugeben.\\
	Welche Temperatur hat der Milchkaffee in beiden Fällen?
	\textit{(Hinweis: Die Temperatur der Mischung ist der Mittelwert der einzelnen Temperaturen: $T=\frac{(T_1+T_2)}{2}$.)}
		\end{itemize}
	\end{enumerate}

	\begin{lsg}{}
		\begin{enumerate}
			\item
			\item
			\item $0,208^\circ\mathrm{C} * \mathrm{min}^{-1}$\\
			Erst abkühlen, dann Milch dazu: $33,44^\circ\mathrm{C}$
			Variante 2: $37,14^\circ\mathrm{C}$
		\end{enumerate}
	\end{lsg}

	\aufgabe{LGS}
	%Nutzen in {alignat}{5}:{NR}{K1}{K2}{K3}{Ergebnis}
	\newcommand{\lgslinethree}[4]{#1 x\quad &+&  #2 y\quad &+&  #3 z\quad & = #4 \\}
	\newcommand{\lgslinetwo}[3]{#1 x\quad &+&  #2 y\quad & = #3 \\}
	Berechnen Sie die Lösungsmenge $\LM\subseteq\R$ der linearen Gleichungssysteme:
	\begin{multicols}{2}
		\begin{enumerate}
			\item
			\begin{alignat*}{5}
				\lgslinethree{}{3}{}{1}
				\lgslinethree{3}{9}{4}{5}
				\lgslinethree{}{3}{2}{3}
			\end{alignat*}
			\item
			\begin{alignat*}{4}
				\lgslinethree{6}{}{}{31}
				\lgslinethree{3}{1}{4\frac 1 2}{19}
				\lgslinethree{-2}{-1}{0}{-15}
			\end{alignat*}
			\item
			\begin{alignat*}{4}
				\lgslinetwo{3}{2}{1}
				\lgslinetwo{1}{-2}{11}
				\lgslinetwo{-2}{}{5}
			\end{alignat*}
		\end{enumerate}
	\end{multicols}



	\aufgabe{Vektorrechnung}
	\begin{enumerate}
		\item Berechne das Skalarprodukt der beiden Vektoren:
		\item Berechne $\alpha$ so, dass die beiden Vektoren orthogonal zueinander sind.
		\item
	\end{enumerate}


\end{document}
