\documentclass[a4paper, oneside]{article}
\usepackage[utf8]{inputenc}
\usepackage[ngerman]{babel}
\usepackage[top=2.5cm, bottom=3cm, outer=2.5cm, inner=2.5cm, heightrounded]{geometry}
\usepackage{graphicx}
\usepackage{morefloats}
\usepackage{wrapfig}
\usepackage{hyperref}
\usepackage{cite}
\usepackage{siunitx}
\usepackage[default]{sourcesanspro}
\usepackage[T1]{fontenc}
\usepackage{url}
\usepackage{marginnote}
\usepackage[font=footnotesize]{caption}
\usepackage{color}
\usepackage{xcolor}
\usepackage{multicol}
\usepackage[fleqn]{mathtools}
\usepackage{amssymb}
\usepackage{wrapfig}
\usepackage[noindentafter]{titlesec}
\usepackage{fancyhdr}
\usepackage{lastpage}
\usepackage{comment}

%% LÖSUNGEN ANZEIGEN
\newif\ifshow
%\showtrue
\showfalse

%%%SECTIONING
\renewcommand*{\marginfont}{\noindent\rule{0pt}{0.7\baselineskip}\footnotesize}

\newcommand{\aufgabe}[1]{\subsection{#1}}
\newcommand{\loesung}[1]{\subsubsection{#1}}

\newcommand{\simpleset}[1]{\ensuremath \left\{ #1 \right\}}
\newcommand{\ematrix}[2]{\renewcommand{\arraystretch}{1}\ensuremath\left(\begin{array}{@{}#1@{}}#2\end{array}\right)}

\renewcommand{\theenumi}{\alph{enumi})}
\renewcommand{\labelenumi}{\text{\theenumi}}

\newcounter{aufgabe}
%\newenvironment{lsg}{\loesung}{}
\ifshow
  \newenvironment{lsg}{\loesung}{}
\else
  \excludecomment{lsg}
\fi

\newenvironment{inhalt}
  {\paragraph{Inhalt des Übungsblatts:}\itemize\let\origitem\item}
  {\enditemize\vspace{2em}}

\newcommand{\R}{\ensuremath\mathbb{R}}
\newcommand{\N}{\ensuremath\mathbb{N}}
\newcommand{\Z}{\ensuremath\mathbb{Z}}
\newcommand{\LM}{\ensuremath\mathbb{L}}
\newcommand{\intd}{\ensuremath\mathrm{d}}
\newcommand{\e}{\ensuremath\mathrm{e}}
\renewcommand{\d}{\,\mathrm{d}}
\newcommand{\stf}[1]{\ensuremath \left[ #1 \right]}

\newcommand{\cas}{\hfill (CAS)}
\newcommand{\seite}[1]{\textit{(S. #1)}}

\newcommand{\vektor}[1]{\ensuremath\begin{pmatrix} #1 \end{pmatrix}}


\everymath{\displaystyle}

%Malpunkte
\mathcode`\*="8000
{\catcode`\*\active\gdef*{\cdot}}

%SECTION
\titleformat{\section}
{\clearpage\setcounter{aufgabe}{0}\vspace{1em}\Large\raggedright\bfseries}
{}
{0pt}
{}

\titleformat{\subsection}[runin]
{\stepcounter{aufgabe}\vspace{1px}\normalfont\raggedright\bfseries}
{A\theaufgabe: }
{0pt}
{\ }

\titleformat{\subsubsection}[runin]
{\normalfont\raggedright\bfseries}
{Lösung \theaufgabe: }
{0pt}
{\ }


%FANCYHDR
\pagestyle{fancy}
\lhead{\small Simon König\\ Joshua Fabian}
\rhead{\small Mathecrashkurs 2018}
\cfoot{Seite \thepage\thinspace von\thinspace\pageref{LastPage}}
\lfoot{}
\renewcommand{\headrulewidth}{0.5pt}
\renewcommand{\footrulewidth}{0pt}

\title{Mathe-Crashkurs 2018 - Übungsblatt}
\date{\today}
\author{Simon König, Joshua Fabian}


\usepackage{xfrac}
\chead{Vorrechnen}
\everymath{\displaystyle}

\begin{document}
\section{Montag}
\subsection{Ableiten}

\begin{equation*}
  f(x)=(\e^{3x}+7)^5\quad f'(x)=5(\e^{3x}+7)^4*3\e^{3x}
\end{equation*}

\begin{equation*}
  f(x)=\sqrt[7]{\cos(-3x)+2} \quad f'(x)=\frac 3 7(\cos(-3x)+2)^{-\sfrac{6}{7}}*\sin(-3x)
\end{equation*}







\clearpage
\vspace*{5em}
\renewcommand*{\arraystretch}{1.4}
\begin{tabular}{r|l|l}
								& \bf Differenzialgleichung & \bf Funktionsterm\\
	\bf linear 				& $f'(t)=k$ 						&	$f(t)=k*t+c$ \\
	\bf exponentiell 	& $f'(t)=b*f(t)$ 				& $f(t)=f(0)*\e^{\ln(b)*t}$\\
	\bf beschränkt 		& $f'(t)=b*(S-f(t))$ 		& $f(t)=S-(S-f(0))*\e^{-\ln(b)*t}$ \\
\end{tabular}
\vspace{3em3}
\begin{description}
	\item[$b$] Wachstumsfaktor
	\item[$k$] $=\ln(b)$ Wachstumskonstante
	\item[$S$] Schranke
\end{description}

\clearpage
\begin{align*}
	f(t_V)&=2*f(0)\\
	f(0)*\e^{k*t_V}&=2*f(0)\\
	\e^{k*t_V}&=2\\
	k*t_V&=\ln(2)\\
	t_V&=\frac{\ln(2)}{k}
\end{align*}
\begin{align*}
	f(t_H)&=\frac{f(0)}{2}\\
	f(0)*\e^{k*t_H}&=\frac{f(0)}{2}\\
	\e^{k*t_H}&=\frac{1}{2}\\
	k*t_H&=\ln\left(\frac12\right)\\
	k*t_H&=\ln(1)-\ln(2)\\
	k*t_H&=0-\ln(2)\\
	t_H&=\frac{-\ln(2)}{k}\\
\end{align*}

\clearpage
\section{Mittwoch}\
Stelle eine Funktionsgleichung für das Wachstum auf und gib die Art des Wachstums an.
\begin{enumerate}
	\item Eine Bakterienkultur besteht anfänglich aus 500 Bakterien. Ihre Anzahl nimmt jede Stunde um 50\% zu.
	\item Ein in trübes Wasser fallender Lichtstrahl wird pro Meter Wassertiefe um 30\% schwächer
	\item Bei einer neuen App erwartet man maximal 30 000 Käufer.
	In einem Modell soll angenommen werden, dass sich die Gesamtzahl der Käufer nach dem Gesetz des beschränkten Wachstums entwickelt.
	Sechs Monate nach Verkaufsbeginn gibt es bereits 20 000 Käufer.
	Bestimmen Sie einen Funktionsterm, welcher die Gesamtzahl der Käufer in Abhängigkeit von der Zeit beschreibt. (vgl. Abi 2017 A1)
\end{enumerate}


\begin{alignat*}{3}
	&8x_1 &+7x_2 &-3x_3 &= 17\\
	&2x_1 &+x_2 &-1x_3 &= 1\\
	&6x_1 &+3x_2 &-2x_3 &= 7\\
\end{alignat*}
\begin{alignat*}{3}
	&10x_1 &+14x_2 &-1x_3 &= -1\\
	&20x_1 &+34x_2 &-3x_3 &= -20\\
	&15x_1 &+21x_2 &-0,5x_3 &= 25,5\\
\end{alignat*}
\begin{alignat*}{3}
	&\frac{16}{3}x_1 &+\frac{8}{3}x_2 &+\frac{8}{3}x_3 &= \frac{88}{3}\\
	&\frac{20}{3}x_1 &+4x_2 &+4x_3 &= 24\\
	&0x_1 &+\frac{4}{3}x_2 &+\frac{4}{3}x_3 &= 4\\
\end{alignat*}



\end{document}
